% I seguenti commenti speciali impostano:
% 1. 
% 2. PDFLaTeX come motore di composizione;
% 3. tesi.tex come documento principale;
% 4. il controllo ortografico italiano per l'editor.

% !TEX encoding = UTF-8
% !TEX TS-program = pdflatex
% !TEX root = tesi.tex
% !TeX spellcheck = <none>

\documentclass[10pt,                    % corpo del font principale
               a4paper,                 % carta A4
               twoside,                 % impagina per fronte-retro
               openright,               % inizio capitoli a destra
               english,                 
               italian,                 
               ]{book}    

\usepackage[utf8]{inputenc}             % codifica di input; anche [latin1] va bene
                                        % NOTA BENE! va accordata con le preferenze dell'editor

%**************************************************************
% Importazione package
%************************************************************** 

%\usepackage{amsmath,amssymb,amsthm}    % matematica

\usepackage[english, italian]{babel}    % per scrivere in italiano e in inglese;
                                        % l'ultima lingua (l'italiano) risulta predefinita

\usepackage{bookmark}                   % segnalibri

\usepackage{caption}                    % didascalie

\usepackage{chngpage,calc}              % centra il frontespizio

\usepackage{csquotes}                   % gestisce automaticamente i caratteri (")

\usepackage{emptypage}                  % pagine vuote senza testatina e piede di pagina

\usepackage{epigraph}					% per epigrafi

\usepackage{eurosym}                    % simbolo dell'euro

\usepackage[T1]{fontenc}                % codifica dei font:
\usepackage{lmodern}                                        % NOTA BENE! richiede una distribuzione *completa* di LaTeX

%\usepackage{indentfirst}               % rientra il primo paragrafo di ogni sezione

\usepackage{graphicx}                   % immagini

\usepackage{hyperref}                   % collegamenti ipertestuali



\usepackage[binding=5mm]{layaureo}      % margini ottimizzati per l'A4; rilegatura di 5 mm

\usepackage{listings}                   % codici

\usepackage{microtype}                  % microtipografia

\usepackage{mparhack,fixltx2e,relsize}  % finezze tipografiche

\usepackage{nameref}                    % visualizza nome dei riferimenti                                      

\usepackage[font=small]{quoting}        % citazioni

\usepackage{subfig}                     % sottofigure, sottotabelle

\usepackage[italian]{varioref}          % riferimenti completi della pagina

\usepackage[dvipsnames,table]{xcolor}         % colori

\usepackage{pgf-pie}
\usepackage{tikz}
\usepackage{pgfplots}
\usepackage{makeidx}
\usepackage{caption}
\usepackage{colortbl}
\usepackage{multirow}
\usepackage{color}
\usepackage{booktabs}                   % tabelle                                       
\usepackage{tabularx}                   % tabelle di larghezza prefissata                                    
\usepackage{longtable}                  % tabelle su più pagine                                        
\usepackage{ltxtable}                   % tabelle su più pagine e adattabili in larghezza

\usepackage[toc, acronym]{glossaries}   % glossario
                                        % per includerlo nel documento bisogna:
                                        % 1. compilare una prima volta tesi.tex;
                                        % 2. eseguire: makeindex -s tesi.ist -t tesi.glg -o tesi.gls tesi.glo
                                        % 3. eseguire: makeindex -s tesi.ist -t tesi.alg -o tesi.acr tesi.acn
                                        % 4. compilare due volte tesi.tex.

\usepackage[backend=biber,style=verbose-ibid,hyperref,backref]{biblatex}
                                        % eccellente pacchetto per la bibliografia; 
                                        % produce uno stile di citazione autore-anno; 
                                        % lo stile "numeric-comp" produce riferimenti numerici
                                        % per includerlo nel documento bisogna:
                                        % 1. compilare una prima volta tesi.tex;
                                        % 2. eseguire: biber tesi
                                        % 3. compilare ancora tesi.tex.

\definecolor{red}{rgb}{1.0, 0.0, 0.0} %http://latexcolor.com/
\definecolor{blue-violet}{rgb}{0.54, 0.17, 0.89}
\definecolor{blush}{rgb}{0.87, 0.36, 0.51}
\definecolor{glaucous}{rgb}{0.38, 0.51, 0.71}
\definecolor{lightcornflowerblue}{rgb}{0.6, 0.81, 0.93}
\definecolor{moonstoneblue}{rgb}{0.45, 0.66, 0.76}
\definecolor{darkcandyapplered}{rgb}{0.64, 0.0, 0.0}
\definecolor{green}{rgb}{0.01, 0.75, 0.24}
\definecolor{coquelicot}{rgb}{1.0, 0.22, 0.0}
\definecolor{dbscan}{HTML}{ABDDD4}
\definecolor{constructCluster}{HTML}{2B83BA}
\definecolor{storeClusterComponent}{rgb}{0.91, 0.41, 0.17}
\definecolor{light-gray}{gray}{0.50}
\definecolor{candyapplered}{rgb}{1.0, 0.03, 0.0}
\definecolor{amber}{rgb}{1.0, 0.75, 0.0}
\definecolor{burntorange}{rgb}{0.8, 0.33, 0.0}
\definecolor{flame}{rgb}{0.89, 0.35, 0.13}
\definecolor{lava}{rgb}{0.81, 0.06, 0.13}
\definecolor{laserlemon}{rgb}{1.0, 1.0, 0.13}
\definecolor{mediumspringgreen}{rgb}{0.0, 0.98, 0.6}
\definecolor{palecopper}{rgb}{0.85, 0.54, 0.4}
\definecolor{alizarin}{rgb}{0.82, 0.1, 0.26}
\definecolor{atomictangerine}{rgb}{1.0, 0.6, 0.4}
\definecolor{ashgrey}{rgb}{0.7, 0.75, 0.71}
\definecolor{beaver}{rgb}{0.62, 0.51, 0.44}
\definecolor{burlywood}{rgb}{0.87, 0.72, 0.53}
\definecolor{cadet}{rgb}{0.33, 0.41, 0.47}
\definecolor{charcoal}{rgb}{0.21, 0.27, 0.31}
\definecolor{darkpastelblue}{rgb}{0.47, 0.62, 0.8}
\definecolor{blue-violet}{rgb}{0.54, 0.17, 0.89}
\definecolor{blush}{rgb}{0.87, 0.36, 0.51}
\definecolor{glaucous}{rgb}{0.38, 0.51, 0.71}
\definecolor{lightcornflowerblue}{rgb}{0.6, 0.81, 0.93}
\definecolor{moonstoneblue}{rgb}{0.45, 0.66, 0.76}




\input{tesi-config}                     % file con le impostazioni personali

\begin{document}
%**************************************************************
% Materiale iniziale
%**************************************************************
\frontmatter
\input{inizio-fine/frontespizio}
% !TEX encoding = UTF-8
% !TEX TS-program = pdflatex
% !TEX root = ../tesi.tex
% !TeX spellcheck = de_DE

%**************************************************************
% Colophon
%**************************************************************
\clearpage
\phantomsection
\thispagestyle{empty}

\hfill

\vfill

\noindent\myName: \textit{\myTitle,}
\myDegree,
\textcopyright\ \myTime.
\input{inizio-fine/sommario}
% !TEX encoding = UTF-8
% !TEX TS-program = pdflatex
% !TEX root = ../tesi.tex
% !TeX spellcheck = de_DE

%**************************************************************
% Indici
%**************************************************************
\cleardoublepage
\pdfbookmark{\contentsname}{tableofcontents}
\setcounter{tocdepth}{4}
\setcounter{secnumdepth}{4}
\tableofcontents
%\markboth{\contentsname}{\contentsname} 
\clearpage

\begingroup 
    \let\clearpage\relax
    \let\cleardoublepage\relax
    \let\cleardoublepage\relax
    %*******************************************************
    % Elenco delle figure
    %*******************************************************    
    \phantomsection
    \pdfbookmark{\listfigurename}{lof}
    \listoffigures

    \vspace*{8ex}

    %*******************************************************
    % Elenco delle tabelle
    %*******************************************************
    \phantomsection
    \pdfbookmark{\listtablename}{lot}
    \listoftables
        
    \vspace*{8ex}
\endgroup

\cleardoublepage

\cleardoublepage

%**************************************************************
% Materiale principale
%**************************************************************
\mainmatter
% !TEX encoding = UTF-8
% !TEX TS-program = pdflatex
% !TEX root = ../tesi.tex
% !TEX spellcheck = it-IT

%**************************************************************
\chapter{L'Azienda}
\label{cap:azienda}
\vspace{20pt}

\section{IKS}

IKS \emph{(Information Knowledge. Supply)} è un'azienda padovana 
fondata, dall'attuale Amministratore Delegato Paolo Pittarello, nel 1999. 

Nell'insieme, IKS unisce figure di alto profilo con lo scopo di 
proporre soluzioni innovative alle richieste di mercato dell'\gls{ict} 
sia italiano che estero. Le soluzioni offerte interessano in particolare 
gli ambiti della sicurezza, dell'infrastruttura e della governance \gls{it}.  

L'azienda è in continua ricerca tecnologica. Investendo sulla formazione 
del proprio personale, IKS si impegna di portare solo valore aggiunto 
al business dei propri clienti. Inoltre, l'Azienda crede fortemente 
nell'innovazione come strumento verso un ambiente digitale comune, 
\gls{agile} e totalmente disponibile. 

Il quartier generale aziendale è a Padova. Inoltre, IKS possiede uffici 
anche nelle seguenti città: Roma, Milano e Trento.

A partire dallo scorso anno (2016), IKS SRL, Kirey SRL, Insirio SPA e 
System Evolution SRL hanno  fondato il Gruppo Kirey. L'obiettivo comune 
delle quattro aziende è l'unione delle competenze complementari e 
garantire un portfolio completo di soluzioni ai clienti attuali e futuri. 

La creazione del Gruppo Kirey è stata guidata dalla Synergo SGR., società di 
\emph{private equity}. Il presidente del nuovo Gruppo commerciale è Vittorio 
Lusvarghi.   

A seguito della creazione del Gruppo, IKS SRL e le restanti tre realtà 
aziendali hanno conservato la propria struttura di governance e management, 
con il fine di garantire la propria continuità gestionale. 

\section{Profilo dell'azienda}
\subsection{Servizi e prodotti offerti}

Nel corso degli anni, IKS si è fatta notare per gli enormi contributi 
innovativi nell'ambito della sicurezza informatica. Tuttavia, essa non 
è limitata a questo ambito. Infatti, gli altri ambiti di applicazione 
sono: infrastruttura e governance IT. 

Di seguito vengono presentati i servizi offerti da IKS per ciascun ambito di 
applicazione:

\begin{itemize}
	\item \textbf{IT Security}\\
	 \begin{itemize}
	 	\item \textbf{Risk analysis e vulnerability assessment}\\ 
	 	È importante garantire la sicurezza dell'infrastruttura 
		informatica nel suo complesso. A questo scopo, IKS offre un 
		servizio orientato alla ricerca di eventuali vulnerabilità e 
		analisi dei rischi a esse collegate;
	 
	 	\begin{figure}[htbp]
	 		\begin{center}
	 			\includegraphics[height=4cm]{risk-management}
	 			\caption{Visione a processo della gestione del 
				 rischio.Immagine tratta da: http://bit.ly/2rh3V0A.}
	 		\end{center}
	 	\end{figure}
	 		 	
		\item \textbf{Audit management}\\
		Le aziende di continuo sono sottoposte a controlli di vario 
		genere; il loro scopo è l'accertamento della 
		regolarità delle aziende con: certificazioni, normative, bilanci 
		ed ecc. IKS offre un servizio di supporto per le aziende con 
		il fine di agevolare le attività di \emph{auditing} 
		ed eventualmente per migliorare i loro processi interni;
		\begin{figure}[htbp]
			\begin{center}
				\hspace{3em}
				\includegraphics[height=3.5cm]{audit-management}
				\caption{Flusso di lavoro durante lo 
				svolgimento di un \emph{audit}.
				Immagine tratta da: http://bit.ly/2rdFhfv.}
			\end{center}
		\end{figure}
	
	 	\item \textbf{Difesa perimetrale}\\ 
	 	Sempre in ambito della sicurezza è importante prendere le 
		giuste misure per garantire a priori uno specifico livello 
		di sicurezza e limitare a zero le intrusioni dall'esterno 
	 	di un'infrastruttura IT aziendale. A questo scopo, IKS offre 
		un'insieme di soluzioni orientare al monitoraggio degli 
		accessi a sistemi aziendali, dei permessi sulle operazioni 
	 	che un utente può effettuare, e molto altro;
	 	\begin{figure}[htbp]
	 		\begin{center}
	 			\includegraphics[height=4cm]{firewall}
	 			\caption{Visione grafica del concetto di difesa 
				perimetrale. Immagine tratta da: 
				http://bit.ly/2s834O2.}
	 		\end{center}
	 	\end{figure}	 	
	 \end{itemize}	
	\item \textbf{IT Infrastructure}\\
	 \begin{itemize}
	 	\item \textbf{Business continuity}\\
	 	In ambito bancario, le infrastrutture informatiche sono molto 
		complesse. La manutenzione delle infrastrutture informatiche non è 
		semplice. La sfida più difficile è garantire che questi sistemi siano 
		operativi al 100\%. Una simile percentuale nella pratica è 
		impossibile. IKS con il proprio gruppo di esperti sono alla 
		continua ricerca di soluzioni per incrementare la percentuale 
		di continuità operativa di questi sistemi. Infatti, le 
		soluzioni offerte dall'azienda sono orientate nel concreto 
		all'infrastruttura del cliente richiedente supporto;
	 	\begin{figure}[htbp]
	 		\begin{center}
	 		\includegraphics[height=5cm]{business-continuity}
				\caption{Visione del ciclo di vita del 
				processo di \emph{business continuity}. 
				Immagine tratta da: http://bit.ly/2qvCmgP.}
			\end{center}
			\end{figure}

%\newpage
		\item \textbf{Virtualization technology}\\
		 Ogni prodotto software di business per portare valore aggiunto 
		 deve essere eseguito. Eseguire un prodotto software per server 
		 fisico richiede la disponibilità di un cospicuo numero di server. 
		 A questo scopo la tecnologia di virtualizzazione permette la 
		 creazione di server virtuali che eseguono programmi e questi 
		 vengono eseguiti da server fisici. I benefici di una simile 
		 infrastruttura è l'ottimizzazione delle risorse di calcolo, 
		 agilità di gestione e sicurezza. Alcune delle soluzioni di 
		 virtualizzazione offerte da IKS sono: VMWare, RHEV ed ecc. 
		 Un'evoluzione della tecnologia di virtualizzazione è il \emph{Cloud}. 
		 In questo ambito, IKS propone soluzioni di migrazione e supporto 
		 verso il Cloud dell'infrastruttura IT classica di un'azienda;  
		 \begin{figure}[htbp]
			\begin{center}
				\includegraphics[height=4cm]{virtualization}
				\caption{Vista a confronto: ambiente 
				server bare metal e virtualizzato. 
				Immagine tratta da: http://bit.ly/2qvtLLk.}
			\end{center}
		 \end{figure}
 	\end{itemize}

	\item \textbf{IT Governance}\\
	\begin{itemize}
		\item \textbf{Service management}\\
		Un servizio informatico di business, a causa della sua criticità,
		richiede costante attenzione. Il monitoraggio del servizio 
		informatico presenta la necessità di enormi investimenti 
		economici. IKS offre piani di gestione per soddisfare 
		anche i più esigenti clienti;
	    
	    \begin{figure}[htbp]
	    	\begin{center}
	    		\includegraphics[height=5cm]{itil}
	    		\caption{Visione della gestione di servizio in 
				prospettiva del \gls{framework} ITIL. Immagine tratta da: 
				http://bit.ly/2qvNryk.}
	    	\end{center}
	    \end{figure}
	    
		\item \textbf{Application and performance monitoring}\\ 
		Ogni prodotto software ha il proprio specifico ciclo di vita. 
		Concluso il ciclo di sviluppo, il prodotto è rilasciato in 
		produzione. La seconda parte del ciclo di vita di un prodotto 
		software è la manutenzione. Il monitoraggio di un applicativo 
		è importante per avere una costante visione dello stato del 
		prodotto e prevenire eventuali esigenze di manutenzione generica 
		oppure di basso profilo a livello di codice sorgente. In questo 
		dominio, grazie a partnership strategiche, IKS offre soluzioni 
		mirate a garantire la miglior possibile esperienza di 
		monitoraggio applicativo;
		
		\item \textbf{System and networking management}\\
		Gestire sistemi e reti informatiche è un compito complesso. 
		L'utilizzo di strumenti adeguati permette di semplificare il 
		lavoro e garantisce un stato consistente del sistema nel tempo. 
		Le soluzioni che IKS offre sono orientate alla flessibilità e 
		facilità d'uso dei prodotti offerti in questo contesto;
	\end{itemize} 
	\item \textbf{Innovation \& Project} \\
	\begin{itemize}
		\item \textbf{Architetture applicative distribuite}\\
		I sistemi informatici diventano sempre più di natura 
		distribuita. IKS offre in questo ambito soluzioni architetturali 
		orientate a microservizi, utilizzando le ultime tecnologie 
		orientate alla containerizzazione e orchestrazione di container; 
		
		\item \textbf{Sviluppo di applicazioni cloud native}\\
		È comune sentire parlare di cloud. Le classiche 
        architetture applicative non riescono a beneficiare della 
		flessibilità del cloud, perché in organizzazione e struttura 
		non sono scalabili e sono difficilmente modularizzabili. Applicazioni
		che vengono gestite nel complesso come un'unica unità prendono il nome di 
		monolite. La diretta conseguenza di una simile organizzazione è 
		il carattere statico e poco flessibile dell'applicazione. Paradigmi nuovi, 
		per la messa in esercizio di applicazioni, mancano l'intigrazione 
		con architetture software tradizionali. Per questo motivO, 
		le applicazioni devono essere sviluppate fin dal principio con 
		un'architettura orientata al Cloud. Una buona guida, di sviluppo 
		di applicazioni orientate al Cloud, è la segeunte: \textit{Twelve Factor-Factor App}.,
		Heroku, servizio  PaaS per applicazioni cloud native, promuove 
		continuamente l'importanza dei 12 principi alla base della filosofia 
		\emph{cloud native}. In questa direzione IKS propone servizi di sviluppo 
		di applicazioni di business orientate all'affidabilità, resilienza, 
		scalabilità orizzontale ed ecc.
		
		\begin{figure}[htbp]
			\begin{center}		
			\includegraphics[height=5cm]{monolith-microservice}
			\caption{Visione architetturale a monolite e 
				microservizi a confronto. 
				Immagine tratta da: http://bit.ly/2rh1niY.}
			\end{center}
			\end{figure}
			\end{itemize} 
		\end{itemize}

La clientela tipica di IKS sono le aziende operanti nei seguenti ambiti: 
pubblica amministrazione, bancario, assicurativo e servizi. 

Una lista dettagliata delle referenze può essere consultata sul sito
di IKS (\url{https://www.iks.it/referenze.html}).

\subsection{Struttura organizzativa}

Ad oggi, IKS conta più di 100 dipendenti. La sua organizzazione interna è 
riassunta nel diagramma in \textbf{Figura 1.8}.

In seguito, descrivo le unità operative che costituiscono il nucleo 
decisionale dell'azienda. Queste unità sono:
\begin{itemize}
	\item \textbf{Direzione}\\ 
	Definisce gli orientamenti e le politiche aziendali, gli 
	obiettivi per la qualità, riesamina periodicamente il sistema di 
	qualità e gestisce il piano di formazione dei dipendenti in funzione 
	alle esigenze e motivazioni personali;
	\item \textbf{Direzione Commerciale}\\
	Definisce le politiche commerciali, gli obiettivi e le risorse 
	necessarie. Promuove i servizi e prodotti dell'azienda. Gestisce i 
	clienti, i fornitori e le offerte contrattuali;
	\item \textbf{Direzione tecnica o Operation}\\
	Supporta la Direzione Commerciale nella valutazione commerciale di 
	prodotti e/o offerte dal punti di vista tecnico. Gestisce a livello 
	tecnico i progetti e servizi. Pianifica le risorse necessarie per 
	i prodotti/servizi. Verifica lo stato del prodotto/servizio offerto;	
	\item \textbf{Amministrazione \& Finanza}\\
	Gestisce la documentazione di progetto, su coordinamento della direzione 
	commerciale e tecnica. Gestisce l'archiviazione della documentazione;
	\item \textbf{Acquisti}\\
	Su coordinamento della Direzione, gestisce i fornitori di prodotti e 
	servizi. Gestisce il processo di acquisizione di nuovi prodotti o 
	servizi. Il processo di acquisizione è guidato dalle necessità 
	interne aziendali oppure da quelle dei clienti;
	\item \textbf{Assicurazione Qualità}\\
	È a stretto contatto solo con la Direzione. Gestisce il piano di 
    qualità, coordina le attività di ispezione, misura e stima il livello 
	della qualità aziendale;
	\item \textbf{Business Unit (BU)}\\
	Gestisce i progetti o servizi concordati con il Cliente. Rendiconta 
	direttamente alla Direzione Tecnica e gestisce l'emissione delle 
	fatture verso il Cliente. 
\end{itemize}


\begin{figure}[htbp]
	\begin{center}
		\includegraphics[height=8cm]{organigramma}
		\caption{Organigramma aziendale}
	\end{center}
\end{figure}



\subsection{Processi aziendali}

IKS, a partire dal 2003, è certificata UNI EN ISO 9001. Questo certifica che 
l'azienda cura molto la qualità del proprio lavoro. Infatti, il miglioramento 
continuo permette all'azienda di rimanere competitiva e consolidare 
la propria posizione di leader sul mercato del ICT italiano. Riporto, di seguito, 
alcuni obiettivi di qualità dell'azienda:

\begin{itemize}
	\item Mantenere e aumentare il livello di soddisfazione del Cliente;
	\item Operare in modo efficiente ed efficace per soddisfare i requisiti 
		  contrattuali, norme e regolamenti;
	\item Monitorare i propri processi per: garantire azioni correttive 
	      tempestivamente e permettere un comportamento pro attivo, 
	      anticipare i bisogni e predire le risorse aziendali necessarie 
	      prima dell'effettivo bisogno; 
	\item Assicurare una adeguata formazione al Personale.
\end{itemize}


Il Cliente copre un ruolo importante nella quotidinità di IKS. Infatti, 
l'azienda cerca di coinvolgere i propri clienti il più possibile., questo 
è necessario per comprendere meglio i bisogni attuali del cliente e 
cogliere esigenze future. 
In azienda, il passo successivo alla formalizzazione del bisogno del 
Cliente segue un'attività di analisi dei requisiti. L'obiettivo 
dell'attività è la dettagliata comprensione del contesto applicativo, 
quali sono le parti interagenti e quali possono essere i rischi, durante 
l'attività di progetto, per implementare i bisogni del Cliente.

A progetto concluso, il Cliente valuta criticamente la soluzione 
presentata. La valutazione, eventualmente, coinvolge un reclamo. Questo 
è rivolto alla Direzione dell'Azienda.

IKS organizza il proprio lavoro per processi: primari, direttivi e di supporto. 
Ciascuna categoria di processo definisce delle responsabilità e compiti. 
Per esempio, i processi organizzativi interessano le attività per: definire la politica 
e strategia aziendale, pianificare e allocare le risorse, riesaminare la gestione del 
sistema di qualità. Invece, i processi primari ricoprono attività che garantiscono 
un diretto ricavo economico per l'azienda e danno un valore aggiunto al prodotto o 
servizio fornito. Esempi di attività in questa categoria sono: proporre offerte 
commerciali ai clienti, progettare e sviluppare prodotti software, erogare servizi IT. 
L'ultima categoria di processi sono i processi di supporto. Le consone attività giornaliere 
riguardano: gestire le risorse umane, l'infrastruttura 
e gli ambienti di lavoro., monitorare e analizzare la qualità aziendale. 

In \textbf{Figura 1.9} segue una presentazione dello schema organizzativo 
utilizzato dall'Azienda, durante il ciclo di vita di un progetto.

Periodicamente, il responsabile della qualità, incaricato della Direzione, attua 
attività d'ispezione. L'obiettivo dell'attività è il controllo del livello di qualità
fornita dai dipendenti aziendali. A posteriori, segue un'attività di 
analisi e misura dei livelli di qualità organizzati per BU, servizio e 
prodotto offerto da IKS.

\begin{figure}[htbp]
	\begin{center}
		\includegraphics[height=7cm]{relazione-responsabilita}
		\caption{Rappresentazione grafica del coinvolgimento del 
		Cliente e i corrispettivi livelli degli interventi del gruppo 
		commerciale, tecnico, direzionale e di supporto nella gestione 
		di un'offerta di progetto.}
	\end{center}
\end{figure}


\section{Rapporto con l'innovazione}
L'innovazione è il processo di gestione dell'intero ciclo di vita di un'idea. 
L'obiettivo è: portare un miglioramento di processo aziendale, di prodotto e/o 
di servizio. Le conseguenze dirette del miglioramento sono: valore aggiunto, per 
l'azienda, in termini di rientro economico e soddisfare un 
bisogno, per il Cliente, in modo efficace ed efficiente. 

L'approccio innovativo induce l'utilizzo dell'informazione, della creatività e 
dello spirito d'iniziativa per raccogliere maggior valore aggiunto dalle risorse a 
disposizione. L'azienda utilizza l'innovazione per soddisfare in modo pro attivo 
le richieste del Cliente. Questo principio è pienamente il linea con la strategia 
di qualità aziendale: \textit{client first}.

La modalità di innovazione di IKS è un approccio incrementale. Inizialmente 
l'azienda cerca di soddisfare i bisogni principali e raggiungere il prima 
possibile gli obiettivi minimi pre-fissati. In seguito, l'azienda migliora 
la propria offerta mediante incrementi continuativi di dettaglio. 

Per supportare l'innovazione, IKS ha creato una cultura aziendale che permette 
ai propri dipendenti di scambiarsi idee, sperimentare, imparare in gruppo e mettere in 
atto la propria creatività. Non manca la comunicazione con i propri responsabili. 
Questi sono i primi a motivare di continuo le risorse umane a loro disposizione. 
Il dialogo dipendente-responsabile non è verticale. La cultura aziendale in questa 
direzione è molto drastica: favorire uno scambio di idee in modo che esso sia  
equo, semplice e non orientato alle gerarchie aziendali. 

In questo contesto, per l'intera durata del mio periodo di stage e dopo un 
primo momento di ambientamento, io ho beneficiato molto del clima aziendale. 
Infatti, non è mancato il libero confronto con il tutor aziendale, il quale ha  
mostrato disponibilità e apertura al mio spirito d'iniziativa. Sempre mio tutor 
aziendale ha supportato me in ogni scelta decisionale che io abbia motivato e 
ritenuto significativa per il beneficio del mio progetto. 

Le idee sono una parte del processo di gestione dell'innovazione: la realtà è 
molto più complessa. IKS non possiede un effettivo processo di gestione a 
livello aziendale. Questo viene gestito da un gruppo di persone con competenze 
trasversali e a livelli organizzativi differenti. 

\begin{figure}[htbp]
   \begin{center}
	\includegraphics[height=5cm]{innovation-management}
	\caption{Il legame attivo tra innovazione e il processo del cambiamento 
	e gestione della conoscenza. Immagine tratta da: http://bit.ly/2qty3XD.}
   \end{center}
\end{figure}

\newpage              
% !TEX encoding = UTF-8
% !TEX TS-program = pdflatex
% !TEX root = ../tesi.tex
% !TEX spellcheck = it-IT

%**************************************************************
\chapter{L'azienda e gli stage}
\label{cap:stage}
%**************************************************************

\intro{Il presente capitolo è dedicato al rapporto dell'azienda 
	con gli stage in collaborazione con l'Università degli Studi di Padova. Nello 
	specifico presento: l'ambito, le motivazioni, obiettivi 
	aziendali e personali, e i vincoli del mio progetto di stage.
}

%**************************************************************
\vspace{20pt}
Lo spirito dell'azienda è investire costantemente sui propri dipendenti e 
soddisfare i bisogni tecnologici del mercato con le giuste competenze. 
I dipendenti tecnici dedicano parte della propria giornata di lavoro in 
approfondimenti tecnologici e attività di laboratorio. 

A supporto dell'attività di laboratorio è stato installato un ambiente virtuale. 
Avere a disposizione un simile environment permette di sperimentare con: nuove 
tecnologie, integrazione di sistemi ed ecc. Inoltre, permette di mettere a 
disposizione degli stagisti un infrastruttura dedicata ai loro progetti di 
stage. Oltre alla virtualizzazione è possibile sperimentare nei 
seguenti ambiti: storaging, networking e security. Temi comuni 
di approfondimento specializzante sono: Cloud, Machine Learning e Analytics. 

Di recente, l'Azienda ha concluso una partnership strategica con AWS, il provider
di cloud pubblico. L'accordo di collaborazione permetterà a IKS di portare 
molte delle proprie soluzioni sul Cloud e beneficiare delle sue 
peculiarità: elasticità, flessibilità nella gestione di infrastrutture ed ecc. 

Per monitorare e prevenire le frodi bancarie IKS ha sviluppato alcune soluzioni. 
Queste usano tecniche sofisticate di Machine Learning e analisi dei log. 
Inoltre, è in corso un progetto per il loro rilascio come servizio. 
In questo modo gli utilizzatori avranno la possibilità di integrare la soluzione 
nei propri sistemi e monitorare il flusso delle transazioni bancarie in tempo
reale.  

Ogni anno l'azienda partecipa a StageIT: un evento completamente dedicato agli 
studenti universitari dei Corsi di Laurea in Scienze e Ingegneria Informatica.
Infatti, quest'evento è un'opportunità per lo studente di mettersi in contatto con 
le realtà aziendali e per queste ultime di conoscere i talent più da vicino. 
Durante l'evento è previsto anche un concorso che promuove il miglior progetto 
di stage. In generale su un'insieme di progetti effettuati nell'edizione
precedente dell'evento vengono scelti un sotto insieme ristretto di finalisti
da una commissione orientata alla promozione dell'innovazione. 

Il vincitore, invece, è scelto dagli studenti in tempo reale durante l'evento. Il premio del vincitore è un buono d'acquisto del valore di 500 Euro.

\section{Il valore aggiunto di uno stagista}

IKS è un partecipante attivo a StageIT e annualmente propone fino a 
6 progetti di stage. Questi non sono verticali su un'unica 
tematica ma usualmente coinvolgono temi come: 

\begin{itemize}
	\item Sviluppo di applicazioni basate su web, \gls{cloud}, mobile 
	      o migrazione su \gls{cloud}/mobile di applicazioni tradizionali;
	\item Progettazione di ambienti, metodologie e strumenti di 
	      sviluppo software.
\end{itemize}

Lo stagista è una risorsa importante per l'azienda. Esso viene visto 
come un portatore di novità. In principio, lo stagista è impiegato su progetti 
di sperimentazione. I quali hanno come obiettivo: lo studio e l'analisi 
di fattibilità dell'integrazione delle soluzioni nell'offerta commerciale dell'azienda. 

Per l'intera durata dello stage, lo stagista si emerge in un ambiente di lavoro il più possibile vero simile alla realtà aziendale. Questo permette al tutor esterno di analizzare più da vicino il candidato in stage. E alla fine dello stage allo stagista può essere proposta un'offerta d'assunzione. 

L'Azienda, grazie al contributo degli stagisti, si allinea con i temi di ricerca 
universitari e con le tendenze tecnologiche del momento sul mercato internazionale.

\section{Alcuni temi di stage}
\subsection{AIOps e Machine Learning}
Il progetto di stage tratta l'integrazione del Machine Learning con 
strumenti di Application Performance Monitoring. L'obiettivo dello 
stage è sperimentare integrando diverse soluzioni in questo ambito e 
studiarne il prodotto finale. Una conseguenza critica di questo progetto 
è lo sviluppo di un pensiero critico per affrontare le più difficili sfide 
del monitoraggio di applicazioni e infrastrutture. 

Il presente progetto si colloca in ambito del \textit{application and performance monitoring} che è un servizio offerto dall'azienda al supporto della governance IT.


\subsection{DevOps Automazione}
L'automazione è fondamento di ogni realtà aziendale contemporanea. Infatti, 
il numero di macchine da gestire spesso non è piccolo. Per semplificare i 
compiti di gestione si devono utilizzare strumenti di configurazione e
automazione. Queste tecnologie permettono di automatizzare tutte le operazioni 
manuali che un sistemista spesso compie durante le attività di manutenzione 
giornaliere. L'obiettivo di questo progetto è l'integrazione di alcuni strumenti che semplificano il \gls{patching} dei server e sperimentare con nuove tecnologie del settore.
Il presente progetto si colloca nell'ambito del \textit{system management}. 

\subsection{Sviluppo moduli evolutivi in ambito antifrode}
IKS ha grande esperienza in ambito della sicurezza informatica bancaria. 
Come prodotto risultate di questa esperienza è SMASH. L'obiettivo dello 
stage è estendere il prodotto con qualche funzionalità di monitoraggio di 
azioni sospette. Oltre allo sviluppo di moduli evolutivi lo stagista ha 
la possibilità di apprendere delle competenze forti nell'ambito della 
sicurezza informatica. 
La presente proposta di stage è un progetto inter business unit dell'azienda. 
Esso si colloca nell'ambito dello sviluppo di prodotti software e della sicurezza
informatica nel settore bancario.


\section{Il progetto proposto}
\subsection{Motivazioni}

E' sempre più comune nelle realtà aziendali l'impiego dell'approccio agile nelle attività di lavoro giornaliero. Infatti, questo approccio promuove la comunicazione in generale e focalizza l'attenzione di tutti i stakeholder sul valore finale, per esempio di un prodotto software oppure di una strategia di mercato, che deve essere garantito. 

Se gli sviluppatori hanno come obiettivo primario lo sviluppo di un prodotto software, i professionisti dell'IT hanno come priorità la sua garanzia di servizio e manutenzione periodica. 

\begin{figure}[htbp]
	\begin{center}
		\includegraphics[height=4cm]{devops-wc}
		\caption{Sia gli sviluppatori che i professionisti IT sono portatori di valore: un feedback che coinvolge ambe le parti è essenziale. Immagine tratta da: http://bit.ly/2rM9lBQ.}
	\end{center}
\end{figure}


Tra i due gruppi esiste un muro di incomprensione. Questo fenomeno è dovuto alla mancanza di comunicazione ed interazione. In caso di eventuali problemi che possono incorrere dopo il rilascio del prodotto software è sola responsabilità dei professionisti IT rimuoverli e riportare il prodotto software in uno stato di operatività non a rischio.

Un simile scenario per un'azienda informatica che deve affrontare un numero elevato di rilasci giornalieri non è accettabile. A questo scopo si è creato un movimento culturale, chiamato DevOps, orientato all'unione degli sviluppatori e sistemisti. L'unione promuove un cambio di mentalità, creazione di nuove competenze e sviluppo di nuovi strumenti che diminuiscano la distanza tra le due realtà. 

Il DevOps ha conseguenze più profonde del semplice cambio culturale. Un'azienda che 
approccia il DevOps affronta un cambiamento interno che ripensa il proprio modello di qualità. I benefici del cambio sono i seguenti: maggiore innovazione, agilità nel cogliere i bisogni di mercato del momento, flessibilità nel gestire il cambiamento e maggior qualità di prodotto e processo.

Una tipica rappresentazione del ciclo di vita DevOps è come segue in figura. 

\begin{figure}[htbp]
	\begin{center}
		\includegraphics[height=4cm]{devops-pipeline}
		\caption{Il DevOps abilità l'automazione del processo di rilascio del software e i cambi dell'infrastruttura IT. Immagine tratta da: http://bit.ly/2rsw9nm.}
	\end{center}
\end{figure}

L'abilità di poter cambiare quando necessario in modo agile è un beneficio notevole 
per le aziende; in questo modo i dipendenti sono impiegati nel creare solo valore aggiunto per l'azienda e il mercato. Invece, la gestione dell'infrastruttura è disciplinata, sistematica e standardizzata. Con un approccio standardizzato e ben strumentalizzato è sempre più difficile individuare server nomadi. Può succedere che in caso di operazioni sofisticate di manutenzione un server scompaia dall'orizzonte di visibilità. Se siamo in ambiente Cloud questo risulta in costi in eccesso. Se siamo in ambiente virtualizzato questo risulta in riduzione della capacità complessiva di calcolo. 

Sebbene il DevOps possiede uno scopo più ampio, il \textit{continuous delivery} è un approccio che promuove l'automazione di tutti i processi che vengono coinvolti 
durante il rilascio di un prodotto software. Il \textit{continuous delivery} permette di abbreviare i tempi di rilascio e aumentarne il loro numero, e migliorare la gestione del cambiamento.

Essere veloci nel \textit{delivery} di un prodotto software non è sufficiente. E' necessario prevedere una \textit{pipeline} di \textit{deploy} per il software che si vuole portare nell'ambiente di produzione. L'ambiente direttamente esposto all'uso dei clienti. Automatizzare questo passaggio implica minor intervento manuale e minor numero di errori e conseguentemente maggior rigore nelle attività complessive coinvolte. 

L'utilizzo di \textit{application container} permette di confezionare le applicazioni in un'unità singola che contiene anche le sue dipendenze. Confezionare in questo modo le applicazioni rende facile lo scambio di artefatti tra i diversi gruppi. Di conseguenza lo stesso artefatto creato da uno sviluppatore verrà consegnato al verificatore che attuerà il \textit{testing} di quel particolare prodotto con una specifica versione e sistemista che ne prevederà il suo successivo rilascio.

Un esempio di strumento di containerizzazione è LXC (Linux Kernel Container), Docker ed ecc. Il primo è un progetto che permette il supporto dei container a livello del kernel Linux. Docker è un'altra soluzione di containerizzazione. Le prime versioni Docker offrivano un insieme di API per interagire con LXC, ora Docker offre una propria soluzione di containerizzazione completamente indipendente da quella Linux. 

Con la containerizzazione segue un'estrema facilità nel gestire le applicazioni durante il loro intero ciclo di vita. 

Il cambio di filosofia è percepibile anche al livello architetturale dei prodotti software. In un ambiente dinamico caratterizzato dall'automazione, verifiche e deploy automatici le classiche architetture software non riescono a beneficiare di questa flessibilità. A questo scopo i microservizi rappresentano uno stile architetturale in sintonia con la filosofia delle \textit{pipeline} Unix: ogni microservizio implementa una sola funzionalità - basso accoppiamento.

\begin{figure}[htbp]
	\begin{center}
		\includegraphics[height=5cm]{microservice-example}
		\caption{Esempio di un sistema a microservizi. Immagine tratta da: http://bit.ly/2qNXKxj.}
	\end{center}
\end{figure}

In figura è possibile notare diversi microservizi. Ciascuno ha una responsabilità
ben definita. Questi possono comunicare tra di loro. Per garantire un alto disaccoppiamento tra i servizi è necessario introdurre un microservizio di servizio, chiamato \textit{Service Discovery}, utilizzato come un DNS (Domain Name System). In questo modo i microservizi possono coesistere nello stesso ambiente e comunicare solo quando necessario senza conoscersi direttamente. Inoltre, i microservizi ottengono un'indipendenza dal luogo di esecuzione. Se un microservizio X in esecuzione su una macchina A migra per eseguire su una macchina B allora un microservizio Y che vuole comunicare con X deve contattare il \textit{Service Discovery} per ottenere l'indirizzo di X. L'effetto che si ottiene è un alto tasso di mobilità dei servizi.  

E' usuale incapsulare un microservizio in un container software. In questo modo 
ogni microservizio diventa un'unità funzionale indipendente e nello stesso ambiente possono coesistere due o più copie dello stesso servizio eliminando l'interferenza dell'uno sull'altro. 
Per aumentare la capacità di robustezza e affidabilità di un microservizio sarà 
sufficiente scalare in orizzontale creando una copia aggiuntiva del microservizio che soffre delle proprietà precedentemente citate. 
Il traffico in ingresso sarà bilanciato con qualche politica di distribuzione del carico ai due microservizi attivi tramite un \textit{Load Balancer}. 

\begin{figure}[htbp]
	\begin{center}
		\includegraphics[height=6cm]{richardson-microservices}
		\caption{I microservizi permettono di scalare orizzontalmente per reggere ai più esigenti carichi di lavoro. Immagine tratta da: http://bit.ly/2qI50LR.}
	\end{center}
\end{figure}

I microservizi semplificano di molto l'applicazione complessiva scomponendo il prodotto in sotto applicazioni indipendenti; complicano l'applicazione nel complesso perché vengono aggiunte componenti nuove e il traffico inter microservizio diventa molto più difficile da gestire.

Si aprono nuove sfide sia per gli sviluppatori che per i sistemisti. 
E queste sfide caratterizzano il contratto di collaborazione tra i due gruppi. 


%%% ------------------------------------------------------------------------------

%%% -------------------------------------------------------------------------------

\subsection{Obiettivi aziendali}

Un team interno di IKS ha sviluppato una soluzione di Executive e Malware Dashboard
basata sullo stack applicativo: Elasticsearch, Logstash e Kibana.
 
L'obiettivo principale del mio progetto di stage è la containerizzazione della 
soluzione precedentemente implementata e garantire l'alta affidabilità delle dashboard. 

Inoltre, tramite l'utilizzo della tecnologia dei container applicativi l'azienda 
ha l'obiettivo di individuare una soluzione architetturale dell'applicativo affinché sia portabile su ambienti come Cloud, macchine virtuali e/o fisiche, e sul portatile dello sviluppatore. 

Oltre alla portabilità dell'applicativo è necessario garantire anche la portabilità dell'infrastruttura che ospita l'applicativo per l'esecuzione.

Con la garanzia di alta portabilità il team di gestione dell'infrastruttura potrà gestire lo stesso ambiente in contesti differenti e affini a scopi diversi in modo univoco. 


\subsection{Obiettivi personali}
Come attività preliminare alla ricerca di un progetto di stage per la Laurea ho 
attuato uno studio individuale di mercato. Lo scopo era capire: tendenze 
tecnologiche, architetturali e metodologiche. Se da un lato le mie ricerche 
hanno cercato di cogliere le novità del momento, dall'altro a livello personale 
queste erano mirate alla ricerca di un contesto in cui potermi applicare e maturare. 

Con il presente progetto gli obiettivi personali erano:

\begin{itemize}
	\item Apprendere conoscenze e competenze in ambito di:
		\begin{itemize}
			\item Virtualizzazione basata sulla tecnologia a container; 
			\item Sistemi distribuiti;
			\item Amministrazione di sistema Linux;
	    \end{itemize}
	\item Acquisire esperienza pratica nella gestione delle reti di calcolatori in ambito dei sistemi, nello specifico le reti definite in modo programmatico per le tecnologie orientate alla containerizzazione; 
	\item Acquisire esperienza nell'analisi, progettazione e implementazione di sistemi orientati ai microservizi;
	\item Famigliarizzare con la piattaforma Kubernetes e i principi del \gls{cloud}.
\end{itemize} 

\section{Piano di lavoro}
\label{sec:piano-di-lavoro}
Il piano di lavoro è stato pianificato per un totale di 300 ore complessive. Il 
contenuto del piano è stato presentato in un documento di cui una copia è stata 
consegnata all'Ufficio degli Stage presso l'Ateneo dell'Università di Padova, 
una seconda copia è stata consegnata al tutor interno e l'ultima copia 
controfirmata dall'ufficio stage dell'Università è stata consegnata all'azienda. 
Il piano è stato strutturato in tre fasi il cui contenuto presento di seguito:
\begin{itemize}
\item Fase 1 - Formazione  (56 ore)
	\begin{itemize}
		\item Docker: la tecnologia per la containerizzazione;
		\item Kubernetes: la tecnologia per l'orchestrazione;
		\item ELK: lo stack applicativo;
		\item Verifiche delle competenze acquisite;
	\end{itemize}
\item Fase 2 - Analisi e progettazione  (56 ore)
	\begin{itemize}
		\item Analisi delle funzionalità della soluzione non containerizzata di dashboard;
		\item Analisi delle modalità di containerizzazione delle componenti;
		\item Analisi delle modalità di \gls{deployment};
		\item Progettazione delle modalità di verifica della non regressione;
		\item Progettazione architetturale della soluzione; 
		\item Progettazione della modalità di \gls{deployment};
		\item Documentazione;
	\end{itemize}
\item Fase 3 - Implementazione  (188 ore)	
	\begin{itemize}
		\item Installazione e configurazione dell'orchestratore;
		\item Implementazione della soluzione in un contesto con e senza orchestratore;
		\item Verifica di non regressione;
		\item Documentazione.
	\end{itemize}
\end{itemize}

\section{Vincoli}
\subsection{Vincoli temporali}
Lo stage è durato 8 settimane per un complessivo di 310 ore di lavoro. Ho lavorato a tempo pieno con il seguente orario: 9.00-18.00. Con la pausa pranzo di 1 ora dalle 12.30 alle 13.30. Come stabilito nel PdL (Piano di Lavoro) le attività sono state strutturate in tre fasi. Ogni fase ha coinvolto attività mirate al raggiungimento di specifici obiettivi. Per maggior dettaglio sul contenuto del PdL riferire la \hyperref[sec:piano-di-lavoro]{sezione Piano di Lavoro}.

\subsection{Vincoli tecnologici}

Fin dal primo giorno di lavoro l'azienda mi ha fornito un portatile dedicato per l'itero periodo di stage. Inoltre, mi è stato vietato di collegare alla rete aziendale qualsiasi dispositivo personale. Inoltre, il portatile di lavoro non poteva essere portato a casa.
Per comunicare internamente sono stati utilizzati strumenti di messaggistica istantanea, come Skype, per comunicazioni informali e la posta elettronica.

Oltre a questo vincolo, a livello tecnologico sono state fissate le seguenti tecnologie:

\begin{itemize}
	\item CentOS7: il sistema operativo installato sulle macchine di laboratorio. CentOS7 è la versione open source di RHEL7 (Red Hat Enterprise Linux versione 7);
	\item Docker: è uno strumento che permette in modo estremamente facile la creazione, il deploy e l'esecuzione di applicazioni utilizzando la tecnologia a container. In questo modo l'attenzione dell'utente è focalizzata su questioni diverse dall'installazione e configurazione dell'applicazione. 
	L'architettura di Docker segue in figura. 
	
	\begin{figure}[htbp]
		\begin{center}
			\includegraphics[height=6cm]{docker-architecture}
			\caption{Le parti costituenti la piattaforma Docker sono: il demone, il client e il registry Docker. Immagine tratta da: http://bit.ly/2rmkt7g.}
		\end{center}
	\end{figure}
	
	Le componenti architetturali costituenti la piattaforma Docker sono il:
	demone, client e registry. Si può notare che l'architettura di alto livello è 
	un architettura client/server. Il server di Docker è il demone che ha la responsabilità di gestione dei container sulla macchina locale. Mentre il client si interfaccia tramite un'interfaccia REST al demone e permette di interagire in modo agile con i container, creare reti virtuali, gestire i dati che devono essere condivisi tra i container e il file system locale della macchina. Infine, il registry di Docker è una repository che può essere pubblica o privata e ha la responsabilità di abilitare la condivisione di immagini utili alla creazione dei container. Come modello mentale, in relazione con il paradigma ad oggetti, è possibile paragonare le immagini Docker a classi che devono essere istanziate per la creazione di oggetti, ovvero container. 
	
	Per favorire il libero scambio di immagini Docker, l'azienda Docker Inc ha 
	messo a disposizione degli utenti un hub completamente gratuito. Nella versione privata di una repository è disponibile il servizio di \textit{scanning} delle immagini per individuare le vulnerabilità di sicurezza. 
	
	Quando si esegue il comando run tramite la CLI di Docker il demone controlla che l'immagine da usare per la creazione del container sia presente in locale. In caso affermativo il container viene creato e messo in esecuzione, altrimenti il demone scarica l'immagine dal registry e al termine del \textit{download} istanzia il container;
	\item Kubernetes: è un sistema open source per automatizzare il deploy, la scalabilità e gestione di applicazioni containerizzate. La tecnologia è un 
	prodotto risultante da 15 anni di esperienza in Google con i container. Essa garantisce la portabilità delle applicazioni e l'indipendenza dall'ambiente fisico di esecuzione.
	
	Kubernetes è un sistema distribuito e il modello architetturale è master/slave.
	
	\begin{figure}[htbp]
		\begin{center}
			\includegraphics[height=6cm]{kube-architecture}
			\caption{Le componenti architetturali di Kubernetes sono: API server, Scheduler, Replication Contoller, Kubelet, Kube proxy, Database Etcd. Immagine tratta da: http://bit.ly/2s4eKUR.}
		\end{center}
	\end{figure}
	
	Il master è un pannello di controllo del sistema K8s (Kubernetes). La sua responsabilità è gestire il \textit{workload} a container. Inoltre, esegue le componenti critiche del sistema: il database chiave valore ad alta consistenza etcd, il gestore delle repliche per la scalabilità orizzantale - replication controller e lo scheduler.  
	
	La componente slave esegue il carico di lavoro. Essa comunica solo con il master e salva le informazioni di servizio tramite il API server nel database etcd. 
	
	Ogni componente master e slave eseguono un agente locale chiamato Kubelet. Il compito dell'agente è quello di collegare le varie componenti e comunicare con il demone Docker. 
	
	Infine, il Kube proxy è la componente che gestisce il traffico di rete dell'intera infrastruttura. 
	
    Kubernetes, essendo un sistema fin dall'inizio pensato per essere componibile 
    si può integrare bene con soluzioni di terzi parti, come per esempio: diverse soluzioni per lo storage, diversi plugin per la rete ed ecc;
	
	\item Elasticsearch, Logstash e Kibana: Le tre componenti sono comunemente conosciute con l'acronimo ELK.  Esse vengono utilizzate insieme come una soluzione open source in progetti che hanno forti esigenze di ricerca e analisi di dati. Elasctisearch è il cuore dello stack applicativo. Esso è un database NoSQL e distribuito implementato in Java. Orientato all'immagazzinamento di dati non strutturati permette di effettuare ricerche complesse impiegando millisecondi contro i secondi necessari utilizzando un classico DB SQL. Kibana, invece, è la componente dello stack che offre la funzionalità di visualizzazione dei dati presenti in Elasticsearch. Una peculiare caratteristica di Kibana è l'interfaccia di creazione di cruscotti. Essendo la soluzione nativa di visualizzazione per Elasticseach, Kibana permette di sfruttare questa integrazione per esprimere ricerche molto complesse e visualizzarle a video tramite effetti grafici accattivanti. Infine, Logstash è la componente di estrazione, trasformazione e caricamento dei dati dalla sorgente in Elasticsearch. Con Logstash risulta semplice filtrare l'informazione utile per l'analisi dei dati ed eliminare il rumore di fondo. Essendo uno strumento implementato in Java offre un insieme ricco di strumenti di terzi parti che arricchiscano ulteriormente il suo insieme di funzionalità. Per esempio, tramite uno plugin esterno è possibile programmare Logstash a interagire con le API del social network Twitter per cercare l'informazione che soddisfa dei particolari criteri di ricerca. 
	Dal punti di vista architetturale la soluzione ELK è flessibile e permette di scalare orizzontalmente in proporzione al carico di lavoro. 	
\end{itemize}

Inizialmente sono state fissate anche le rispettive versioni delle componenti 
sopra citate. Tuttavia, nel corso dello stage ho realizzato che bloccare l'evoluzione di un'infrastruttura può comportare qualche problema nel futuro. A questo scopo ho predisposto un ambiente tollerante agli aggiornamenti e che si auto aggiorna. Durante la personalizzazione dell'ambiente mi sono ispirato al principio  \textit{self driven infrastructure} di CoreOS. 
In questo modo gli aggiornamenti delle componenti possono essere effettuati in modo completamente trasparente.

Le immagini per la creazione di container dovevano essere provenienti solo 
dal repository ufficiale di Docker ed essere le immagini ufficiali.             
% !TEX encoding = UTF-8
% !TEX TS-program = pdflatex
% !TEX root = ../tesi.tex
% !TEX spellcheck = it-IT

\chapter{Lo svolgimento dello stage}
\label{cap:svolgimento-stage}
\vspace{20pt}

\section{Metodo di lavoro}
Le aziende, indifferentemente dal settore, hanno un proprio 
metodo di lavoro. Ovvero, ciascuna azienda, nel corso della 
propria carriera, crea e mette assieme numerose tecniche 
utili al raggiungimento degli obiettivi di mercato. Un 
sottoinsieme di queste aziende impone agli stagisti, come 
vincolo, il proprio metodo di lavoro. Un'ulteriore sottoinsieme 
di aziende, tramite una negoziazione, sceglie quello più 
opportuno e conveniente per lo specifico progetto di stage ed affine 
al tirocinante. Infine, un restante sottoinsieme di aziende
offre libertà e non impone alcun vincolo sul metodo del lavoro. 
IKS, il proponente del mio progetto di stage, ha lasciato a me 
decidere come organizzare il lavoro di dettaglio del PdL.

Per l'intero periodo dell'attività di stage, mi sono ispirato 
a SEMAT (\emph{Software Engineering Methods and Tools}). 
SEMAT è un'iniziativa che promuove la composizione di tecniche 
e favorisce la creazione di metodi personalizzati di lavoro
e su misura per qualsiasi specifico gruppo di persone 
e/o organizzazione. Il concetto fondamentale, dell'iniziativa SEMAT, 
è quello del \emph{alpha}. Un \emph{alpha} rappresenta: un 
punto di vista, un insieme di obiettivi 
e una \emph{checklist} utile per la valutazione dello stato di avanzamento 
del progetto. L'insieme degli \emph{alpha}, che SEMAT promuove, rappresenta
l'essenza di qualsiasi progetto. Oltre all'essenza, SEMAT propone il 
kernel, un vocabolario. Lo scopo del kernel è di agevolare 
la comunicazione tra i membri del gruppo di lavoro. Il kernel può 
essere utilizzato, inoltre, per la comunicazione tra i membri 
di gruppi diversi.

\begin{figure}[htbp]
	\begin{center}
		\includegraphics[height=9cm]{alpha-semat}
		\caption{Rappresentazione, di alto livello, dei legami 
			     sussistenti tra differenti concetti alpha. Immagine tratta 
				 da: http://bit.ly/2g5oV7b.}
	\end{center}
\end{figure}  

\newpage 

La \textbf{Figura 3.1} presenta l'insieme degli alpha e le relazioni 
sussistenti tra gli alpha adiacenti, su proposta di Ivar Jacobson, 
il pioniere dell'iniziativa SEMAT. 
 
I colori, invece, rappresentano le aree di competenza per la valutazione 
dello stato del progetto. Ciascun colore corrisponde a un obiettivo finale; 
questi obiettivi sono: 

\begin{itemize}
	\item Livello di fidelizzazione del cliente - colore verde;
	\item Maturità della soluzione e della rispettiva implementazione - colore giallo;
	\item Il \textit{backlog} di lavoro - colore blue.
\end{itemize}

Un'interpretazione simile della realtà offre una maggior 
tracciabilità dello stato di avanzamento del progetto. 
Per esempio, in \textbf{Figura 3.2}, l'alpha dei requisiti  
offre un ottimo spunto di riflessione sulla maturità e 
stabilità dei requisiti, individuati nella fase di AdR 
(Analisi dei Requisiti). 
Inoltre, esiste per ciascun alpha un insieme di indicatori, 
che guidano, rispetto al piano iniziale di lavoro, l'andamento 
e la qualità del lavoro complessivo. 
Tramite una lista di controllo, SEMAT propone un modo indipendente 
da processi aziendali di valutazione 
del completamento degli obiettivi.

\begin{figure}[htbp]
	\begin{center}
		\includegraphics[height=6cm]{requirements-semat}
		\caption{Ogni concetto alpha è caratterizzato da un 
			     insieme di elementi descrittivi, come: nome, 
			     l'obiettivo e la lista di controllo. Immagine tratta 
				 da: http://bit.ly/2vgPqJf.}
	\end{center}
\end{figure}  

La rappresentazione dei temi frequenti in un progetto 
software tramite gli alpha incrementa la comprensione
generale del gruppo di lavoro.

Per esempio, il alpha dei requisiti è caratterizzato 
dal seguente insieme di obiettivi da raggiungere:

\begin{itemize}
	\item \textit{Conceived}: Dimostrata necessità per un nuovo prodotto;
	\item \textit{Bounded}: Lo scopo del nuovo prodotto sono chiari;
	\item \textit{Coherent}: I requisiti offrono una chiara prospettiva del nuovo prodotto software;
	\item \textit{Acceptable}: I requisiti descrivono il prodotto che è accettato dal cliente;
	\item \textit{Addressed}: Una buona parte dei requisiti sono stati implementati, il sistema 
	      soddisfa la visione che il cliente ha del sistema;
	\item \textit{Fulfilled}: I requisiti che sono stati affrontati sono in accordo con gli 
	      obiettivi pattuiti con il cliente.
\end{itemize} 

Durante l'analisi dei requisiti, ho utilizzato questi obiettivi per valutare
lo stato del mio lavoro, rispetto alla pianificazione iniziale. 
Inoltre, quando ho raggiunto l'obiettivo \textit{"Coherent"} dei requisiti,
ho acquisito una buona visione del sistema da sviluppare nel complesso    
Infatti, la lista di controllo, in allegato a ciascun alpha, mi ha guidato  
nell'attività di consuntivazione. Ho trovato d'aiuto questa lista 
di controllo anche come fonte di requisiti e domande critiche
sul progetto e sistema.   

Ho utilizzato, per esempio, questa griglia per valutare la mia situazione
di progresso rispetto alla pianificazione di periodo. 

%\newpage 

Per un maggior controllo e visibilità del grado di 
completamento del progetto, ho utilizzato l'applicazione "SematAcc", 
che è raggiungibile al seguente indirizzo 
\url{http://sematacc.herokuapp.com/}. 
L'applicazione mi ha permesso, grazie al cruscotto intuitivo 
da essa offerto, di concentrare la mia attenzione su obiettivi di valore 
aggiunto al progetto e temporeggiare con le attività di priorità
più bassa. In questo modo, ho ottenuto un modo agile e semplice 
per gestire i rischi.

A progetto concluso, la situazione del progetto è come in
\textit{Figura 3.3}. I dati, che presento, mostrano uno stato 
salutare per il progetto. Questi dati non rappresentano il massimo
valore; visto che, gli obiettivi sono in continua evoluzione, come 
le tecnologie. 

Ho pianificato e trattato il margine 
rimanente come un margine di evoluzione 
e miglioramento del sistema.

\begin{figure}[htbp]
	\begin{center}
		\includegraphics[height=8cm]{final-semat}
		\caption{Rappresentazione dello stato del progetto, 
			a seguito dell'ultima consegna ufficiale e 
			superamento della revisione di accettazione.}
	\end{center}
\end{figure}  

Aderire all'iniziativa SEMAT e seguire approcci metodologici  
flessibili è vantaggioso. Nel mio caso, ho composto un proprio 
metodo di lavoro; ho ottenuto una semplice gestione dei 
rischi e ho migliorato la mia sensibilità per recepire 
lo stato del progetto in modo naturale ed intuitivo. 
Inoltre, ho migliorato la metodologia di lavoro e, 
con disciplina, ho acquisito le capacità di controllo 
sull'incertezza, in momenti di decisioni importanti. 

%\newpage 

\section{Pianificazione}
Durante lo stage ho lavorato 316 ore. 
Complessivamente, queste ore corrispondono poco 
più di 8 settimane di lavoro effettivo. Nel mio 
caso, lo stage è iniziato il 12/12/2016 e 
terminato il 20/02/2017. In questo arco di 
tempo, è incluso anche il periodo delle 
festività natalizie. Causa assenze impreviste 
da parte mia, nell'ultima settimana di lavoro,
ho esteso la terminazione dello stage dal 
giorno 20/02/2017 al 23/02/2017.

\begin{figure}[htbp]
	\begin{center}
		\includegraphics[height=3cm]{piano-di-lavoro}
		\caption{Piano di lavoro ufficiale.}
	\end{center}
\end{figure}

In \textbf{Figura 3.4} presento il diagramma di Gantt del piano 
di lavoro,  i cui dettagli ho descritto
\hyperref[sec:piano-di-lavoro]{sezione Piano di Lavoro}.  

Il diagramma di Gantt presenta una pianificazione generica del PdL; 
tuttavia, essendo io libero nell'organizzazione del lavoro, 
ho rivisto e riprogrammato le attività del PdL come in \textbf{Figura 3.5}. 

\begin{figure}[htbp]
	\begin{center}
		\includegraphics[height=6cm]{piano-lavoro-dettaglio}
		\caption{Piano di lavoro rivisto e riorganizzato.}
	\end{center}
\end{figure} 

Il progetto prevede la containerizzazione di una soluzione di
"\textit{Malware Dashboard}" e il rilascio del sistema containerizzato
in un ambiente con e senza orchestratore. In relazione con le modalità 
richieste di messa in esercizio del sistema, ho 
organizzato il lavoro complessivo in due mini progetti.
Ho orientato il primo progetto al rilascio del sistema in un contesto 
containerizzato senza orchestratore, qui ho utilizzato unicamente la 
tecnologia Docker, e un secondo progetto orientato 
all'orchestrazione di container, qui ho utilizzato la tecnologia Kubernetes.
Il secondo progetto ha esteso il primo da un punto di visto 
di un incremento migliorativo.

Come parte di riuso tra i due progetti, ho utilizzato 
l'architettura del sistema complessivo e le componenti 
containerizzate. Una simile organizzazione del progetto 
di stage, mi ha permesso di pianificare i momenti di  
\textit{sprint}; uno sprint è un insieme di compiti, 
usualmente, di breve durata e molto intensi di lavoro. Ho scelto
di operare in questo modo perché ho ritenuto importanti i \textit{feedback} 
del tutor aziendale. Inoltre, ho utilizzato questi feedback come una metrica di 
avanzamento, raggiungimento degli obiettivi e maturità della soluzione. Inoltre, 
conoscere il punto di vista del tutor ha garantito stabilità alla soluzione 
per l'implementazione del sistema. 

\section{Attività di formazione}

Ho seguito un metodo attivo di formazione. Ho arricchito le 
attività di studio teorico con attività di laboratorio. 
Durante le sessioni pratiche, ho svolto piccoli 
progetti. Tramite questi piccoli progetti, ho toccato 
con le mani le tecnologie alla base dello stage. 

\subsection{Docker}

Ho imparato la tecnologia, Docker, allo stato dell'arte della 
containerizzazione. Durante la formazione su Docker, ho 
appreso i concetti alla base della tecnologia, come utilizzare 
lo strumento e come pensare i sistemi orientati ai container.

Tramite il frammento di codice, che allego in seguito, illustro 
come estendere un'immagine Docker della componente d'esempio 
Elasticsearch con una estensione di terzi parti; 
questo frammento di codice è chiamato Dockerfile.

\begin{verbatim}
FROM elasticsearch:2.4
MAINTAINER Andrei Petrov <apetrov.ya@yandex.ru>
ADD hq.tar /usr/share/elasticsearch/plugins
\end{verbatim}

Il Dockerfile rappresenta la configurazione 
di componente e le sue dipendenze software. 
In un altro modo, il Dockerfile descrive 
in modo dichiarativo le dipendenze esterne 
del processo. Il Dockerfile è necessario per 
la creazione dell'immagine. Dall'immagine, invece, 
si possono istanziare, possibilmente, infiniti processi. 
Inoltre, un'informazione curiosa sulla chiave \textbf{ADD}
è la seguente: la direttiva permette, in modo automatico,
l'estrazione di file da un archivio con estensione tar.
Questo \textit{hack} è utilissimo e molto amato
dagli esperti di containerizzazione Docker. 
La direttiva è uno \textit{shortcut} che 
permette di ridurre a una sola registrazione sul 
\textit{Augmented File System}, utilizzato da Docker
per la memorizzazione della configurazione dei container. 
Altrimenti, per copiare un archivio da un contesto 
all'altro durante la creazione dell'immagine, devo
copiare l'archivio e successivamente estrarre i dati.
Anche io preferisco utilizzare la direttiva "ADD".

Quando un prodotto software complesso è costituito da 
un insieme di componenti containerizzate, un modello 
mentale utile per la comprensione delle relazioni tra 
le immagini Docker è il modello ad ereditarietà singola 
dei linguaggi di programmazione ad oggetti, come per
esempio il linguaggio di programmazione Java.
Le immagini, come in alcuni casi le classi, 
possono essere organizzate tramite estensioni.
Questo modello favorisce il riciclo delle 
componenti.

Ho utilizzato i Dockerfile come template 
per la creazione di immagini e container di servizi. 
Questo approccio mi ha permesso di codificare in un file, 
da un lato, un insieme di configurazioni parametrizzate;
dall'altro, includere fisicamente nel package
le dipendenze esterne. In seguito alla codifica dei Dockerfile
ho notato un notevole incremento della mia produttività.

La tecnologia Docker ha rivoluzionato il mondo dello 
sviluppo del software. Questa tecnologia ha introdotto 
uno standard per il confezionamento del software.

Un esempio di istanziazione di un container applicativo
Elasticsearch è il seguente:
 
\begin{verbatim}
docker run -d \
-p $(ES_MASTER_CONTAINER_HTTP_PORT):$(ES_MASTER_CONTAINER_HTTP_PORT) \
-p $(ES_MASTER_CONTAINER_TCP_PORT):$(ES_MASTER_CONTAINER_TCP_PORT) \
--name $(MASTER_CONTAINER_NAME) \
--volumes-from $(ES_DATA_CONTAINER_NAME) \
-v limits.conf:/etc/security/limits.conf \
$(ES_DOCKER_IMG):$(ES_DOCKER_IMG_VERSION) \
-Des.node.name="$(MASTER_ES_NODE)" \
-Des.cluster.name="$(ES_CLUSTER_NAME)" \
-Des.node.master="true" \
-Des.node.data="false"  \
-Des.index.number_of_shards=$(ES_SHARDS_NUM) \
-Des.index.number_of_replicas=$(ES_REPLICAS_NUM) \
-Des.network.host=_site_ \
...
-Des.gateway.recover_after_nodes=2 \
-Des.gateway.expected_nodes=2 \
...		 
\end{verbatim}

Qui, tramite la CLI (Command Line Interface) di Docker, creo e 
ed eseguo un container in modalità demone (processo eseguito in background). 
Successivamente, configuro le porte logiche per abilitare la 
comunicazione interprocesso via rete; specifico un \textit{bucket} 
logico per la condivisione dei dati tra il server e il container, 
tramite le opzioni "--volumes-from" e "-v";  fornisco un alias 
mnemonico al container, tramite l'opzione "--name" ed ecc.
Le opzioni con il prefisso "-Des" rappresentano le configurazioni 
per il processo Java di Elasticsearch. In questo frammento di codice
ho omesso alcune parti.
 
Per verificare le mie conoscenze preliminari su container e Docker, 
ho implementato uno script Bash con l'obiettivo di creare un 
cluster di \texttt{N} nodi di Elasticsearch 
in modo automatico ed autonomo.  Il cluster creato è rilasciato 
in modalità locale alla macchina virtuale ospitante.
Su un insieme di \texttt{N} nodi, \texttt{N - M} nodi sono di 
tipologia "data" e il resto sono di tipologia "master". 
Il numero minimo di nodi master è dato dalla 
seguente formula \texttt{M\_min = M/2 + 1}. 
La quantità \texttt{M\_min} specifica il \textit{quorum} 
iniziale di nodi, necessario per la procedura di elezione 
di un nuovo leader del cluster. 

\subsection{Kubernetes}

L'orchestratore è indispensabile per una efficiente gestione 
di una flotta di container. Durante il periodo di formazione, 
ho affrontato diversi problemi legati al presente orchestratore. 
Alcuni di questi problemi riguardano la creazione di un cluster 
Kubernetes in configurazione semplice. Questa configurazione 
interessa la creazione di un cluster formato da nodi master 
e worker. La configurazione del cluster è sprovvista di alta affidabilità. 
Ho effettuato questa scelta causa la difficoltà della procedura 
di installazione e configurazione di un cluster Kubernetes 
in alta affidabilità. Un cluster Kubernetes può essere creato 
tramite strumenti di terzi parti della tecnologia, fornite
come moduli esterni che estendono il prodotto base. 
Un simile strumento, per esempio, è \textbf{kubeadm}. Inoltre, è 
possibile creare un cluster Kubernetes a mano. Quest'ultima modalità 
è macchinosa e lunga. Durante lo stage ho favorito 
la semplicità; con lo strumento \textit{kubeadm} ho effettuato 
il \textit{deployment} di un cluster Kubernetes in 
breve tempo. Lo strumento gestisce in modo
automatico tutti i passi di configurazione iniziale. 
Inoltre, \textit{kubeadm} offre una procedura di distruzione del
cluster.

Ad esempio, tramite il semplice comando 
\begin{verbatim}
kubeadm init
\end{verbatim}
ho installato e configurato un nodo master di Kubernetes. 
Successivamente, con il token generato, ho aggiunto ulteriori 
nodi worker al nodo master esistente. Per aggiungere un nuovo 
nodo worker ho usato, invece, il seguente comando: 
\begin{verbatim}
kubeadm join <token> <indirizzo IP del master>
\end{verbatim}

Durante l'attività di formazione su Kubernetes ho svolto 
attività di studio mirate all'amministrazione dell'infrastruttura 
Kubernetes e all'organizzazione delle risorse containerizzate 
tramite gli oggetti nativi dell'orchestratore, per esempio: 
ReplicationController, Services, Deployments, Job ed ecc., 
affinché l'orchestratore possa gestire i container in esecuzione. 
L'oggetto base di Kubernetes è il Pod. Questa risorsa è una capsula dove al centro  
risiede un container Docker. Un Pod è anche l'unita atomica di schedulazione.

Per la codifica degli oggetti di Kubernetes ho utilizzato dei file.
Questi file sono un modo dichiarativo per codificare l'infrastruttura.
Nella comunità di Kubernetes essi vengono chiamati \textit{manifest}. 

Un esempio di manifest, che ho codificato, è il seguente

\begin{verbatim}
---
apiVersion: v1
kind: Service
metadata:
namespace: elk-dev
name: elasticsearch
labels:
  component: elasticsearch
  role: client
spec:
 selector:
 component: elasticsearch
 role: client
 ports:
 - name: http
   port: 9200
   protocol: TCP
---
apiVersion: v1
kind: ReplicationController
metadata:
  name: es-client
  namespace: elk-dev
  labels:
   component: elasticsearch
   role: client
spec:
 replicas: 1
 ...

\end{verbatim}

Il linguaggio che ho utilizzato per la 
stesura dei manifest di Kubernetes è il linguaggio 
di markup YAML (Yet Another Markup Language). I manifest 
possono essere stesi in due formati: JSON e YAML. Ho utilizzato
il secondo per facilitare la lettura. Il manifest 
allegato è un esempio parziale di 
configurazione in codice di una componente Elasticsearch 
dell'infrastruttura. La prima risorsa è un Service. 
Un Service implementa la funzionalità di \textit{Service Discovery}. 
Infatti, i microservizi utilizzano i nomi dei Service per abilitare
la comunicazione. La seconda risorsa, invece, rappresenta un 
ReplicationController. Il ReplicationController dichiara che il 
numero di copie per lo specifico microservizio deve essere sempre 
pari a uno; lo stato di questa risorsa è gestito dalla componente 
controller di Kubernetes. In caso che, lo stato della
risorsa inizierà a divergere , il controller riporterà allo stato
specificato la risorsa. Questa funzionalità permette, in caso di errore, 
di schedulare nuovamente il Pod e risanare lo stato della componente tramite 
una sequenza di riavvi continui. E in questo caso il controller 
di Kubernetes segnalerà il motivo di errore della risorsa. Questa funzionalità 
si chiama \textit{self-healing}. 
I dati dei pod possono essere salvati su volumi di memorizzazione esterni. 

La specifica di qualsiasi risorsa è costituita 
da quattro parti fondamentali: apiVersion, kind, template e spec. 
La chiave "apiVersion" determina la versione delle API di Kubernetes 
da utilizzare quando dobbiamo specificare una risorsa di tipo "kind".
Ogni chiave-valore presente internamente alla sezione template deve 
essere interpretata come una meta informazione necessaria all'amministrazione 
delle risorse. Infine, il contesto "spec" descrive la specifica per la risorsa.
Le specifiche riguardano la codifica dello stato della risorsa.

La tecnologia Kubernetes offre una macchina a stati per la gestione dei container.
Questo modello a stati, introdotto da Kubernetes, permette diversi problemi 
legati alla qualità dei servizi, alla quantità di risorse 
utilizzate da ciascuna risorsa ed ecc.

\subsection{Clustering di Elasticsearch}
Una delle caratteristiche più importanti di Elasticsearch è
la funzionalità di \textit{clustering}. Questa funzionalità
permette di eseguire in parallelo un insieme di 
processi Elasticsearch sotto uno spazio di nomi
comune. La singola componente, di un cluster, prende 
il nome di nodo. I nodi di un cluster condividono
dati. L'organizzazione dei dati e la politica della 
loro gestione, in modalità cluster,
è complessa. Durante il periodo di formazione 
ho appreso come creare, organizzare i dati e gestire
la componente Elasticsearch in modalità \textit{clustering}.

Approfondito il concetto di clustering in Elasticsearch,
ho implementato degli script per automatizzare e velocizzare
le procedure di manutenzione del cluster Elasticsearch. 

Se da un lato, il clustering incrementa la percentuale di 
\textit{High Availability}, da un altro lato è necessario 
analizzare i requisiti d'uso del cluster. L'analisi d'uso 
è necessaria per comprendere come configurare i livelli di 
ridondanza dei dati e l'organizzazione di partizionamento
degli indici gestiti da Elasticsearch. Nella terminologia 
Elasticsearch un indice è un database. Durante la formazione
su Elasticsearch, ho realizzato che esigenze di scrittura 
possono impattare le performance del cluster se il 
numero delle repliche è elevato. Questo fenomeno è
dovuto alla grande latenza di propagazione delle 
scritture su ciascuna copia dei dati memorizzati.
Al contrario, in caso di esigenze di lettura, ho 
notato che un alto numero di repliche 
migliora di molto i tempi di risposta del cluster 
Elasticsearch.  

\section{Analisi dell'alta affidabilità}
Una delle più importanti garanzie che il settore IT deve garantire 
al business è la continuità operativa dei servizi informatici a 
supporto. I servizi informatici sono complessi e distribuiti.
A causa della loro natura distribuita, i sistemi informatici 
possono essere sottoposti a problemi, per esempio, di rete, 
che degradano la qualità di erogazione di questi servizi.

L'alta affidabilità è un'abilità, del sistema, che permette
di  garantire la continuità operativa, anche se le singole 
componenti non sono in uno stato di buona salute. L'analisi 
dell'alta affidabilità ha l'obiettivo di comprendere 
a fondo i dettagli della proprietà di alta affidabilità
dell'intero sistema da realizzare. 

In questa fase ho individuato due livelli esclusivi che interessano
l'alta affidabilità: un livello fisico/virtualizzato e uno 
logico/containerizzato. Su raccomandazione del mio tutor aziendale, 
ho tralasciato lo studio dell'alta affidabilità 
a livello fisico e mi sono concentrato su quello logico.
A questo livello, ho analizzato quali sono le componenti 
che possono mettere a rischio lo stato complessivo del sistema.

Nella tabella a seguire, riporto le tecnologie utilizzate e per ciascuna 
di esse caratterizzo la tipologia di applicazione e il rispettivo livello 
di supporto per il clustering.

\begin{center}
\begin{tabular}
	{l||p{5cm}||p{5cm}}	
	\arrayrulecolor{white}
	\rowcolor{glaucous}	
	Componente 	&  
	\makebox[4cm][c]{Tipologia} & 	
	\makebox[5cm][c]{Supporto nativo clustering} \\ 
	\rowcolor{lightcornflowerblue}
	Elasticsearch & 
	\makebox[4cm][c]{Applicazione Java} & 
	\makebox[5cm][c]{Si} \\
	\rowcolor{moonstoneblue}
	Kibana & 
	\makebox[4cm][c]{Applicazione Javascript} & 
	\makebox[5cm][c]{No} \\
	\rowcolor{lightcornflowerblue}
	Logstash & 
	\makebox[4cm][c]{Applicazione Java} & 
	\makebox[5cm][c]{No} \\
	\rowcolor{moonstoneblue}
	Nginx & 
	\makebox[4cm][c]{Applicazione C++} & 
	\makebox[5cm][c]{Si} \\
\end{tabular}		  
\end{center}
\captionof{table}{Elenco delle tecnologie utilizzate con il supporto nativo per il clustering} 

Dalla tabella segue che, alcune componenti non 
offrono un supporto nativo per il clustering.
Per garantire la continuità di servizio, ho 
utilizzato il concetto di \textit{redundancy}. 
Tramite l'esecuzione parallela di più 
processi dello stesso tipo, posso garantire
la continuità di erogazione del servizio, anche 
se al più tutti meno uno di questi processi
,all'improvviso, falliscono. Ad alto livello, 
l'utente può, eventualmente, risentire un 
degrado di \textit{performance}. 
Un ulteriore elemento della mia analisi è stata 
la garanzia del basso accoppiamento 
tra le componenti del sistema. 

Come strumento a supporto dell'analisi dell'alta affidabilità, 
ho utilizzato il modello a cubo della scalabilità. 
Il modello offre tre dimensioni X,Y e Z. Ciascuna dimensione 
rappresenta una oggetto di studio. Per esempio, Elasticsearch
è un'applicazione distribuita che offre nativamente 
il supporto per la scalabilità orizzontale e il partizionamento 
dei dati. Una conseguenza importante da notare in Elasticsearch
è il livello della consistenza. Infatti, Elasticsearch è un 
database NoSQL e non garantisce l'alta consistenza. Questo
è il prezzo da pagare quando l'obiettivo del sistema è la scalabilità
su larga scala.

Sebbene la scalabilità implica l'alta affidabilità, 
per esempio tramite la ridondanza, ho ritenuto 
il modello della scalabilità a cubo un concetto 
importante per comprendere le dimensioni 
dell'alta affidabilità del sistema.

\begin{figure}[htbp]
	\begin{center}
		\includegraphics[height=6cm]{scale-cube}
		\caption{Interpretazione grafica del modello a cubo della scalabilità.
		        Immagine tratta da: http://bit.ly/2wPA42R}
	\end{center}
\end{figure}

La dimensione del sistema determina la difficoltà di gestione. 
La difficoltà di gestione incrementa la probabilità
che a regime (sistema operante a tempo pieno) il sistema
possa rivelare un comportamento non atteso. La garanzia di un appropriato
livello di servizio favorisce un accordo equo tra il fornitore del
servizio e l'utilizzatore. Una metrica che ho 
fissato per la qualità di erogazione del servizio è la classe 3 
di alta affidabilità. Ho calcolato questa metrica tramite 
i test di carico e i test di durata. 

La possibilità di valutare a regime la qualità del servizio 
è un obiettivo dei prossimi incrementi evolutivi del progetto. 
Al momento dello stage ho ritenuto questo obiettivo un requisito opzionale.

Per la misura dell'alta affidabilità,
ho utilizzato la seguente scala espressa in tabella.

\begin{center}
\begin{tabular}
		{l||p{5cm}||p{2cm}||p{2cm}}	
		\arrayrulecolor{white}
		\rowcolor{glaucous}	
		Tipo di sistema &  
		\makebox[5cm][c]{Disservizio (minuti per anno) } & 	
		\makebox[2cm][c]{Affidabilità} & 
		\makebox[2cm][c]{Classe} \\ 
		\rowcolor{lightcornflowerblue}
		Non gestito & 
		\makebox[5cm][c]{50000} & 
		\makebox[2cm][c]{90}  & 
		\makebox[2cm][c]{1} \\
		\rowcolor{moonstoneblue}
		Gestito & 
		\makebox[5cm][c]{5000} & 
		\makebox[2cm][c]{99} & 
		\makebox[2cm][c]{2} \\
		\rowcolor{lightcornflowerblue}
		Bene gestito & 
		\makebox[5cm][c]{500} & 
		\makebox[2cm][c]{99.9} & 
		\makebox[2cm][c]{3} \\
		\rowcolor{moonstoneblue}
		Tollerante ai guasti & 
		\makebox[5cm][c]{50} & 
		\makebox[2cm][c]{99.99} & 
		\makebox[2cm][c]{4} \\
		\rowcolor{lightcornflowerblue}
		Altamente affidabile & 
		\makebox[5cm][c]{5} & 
		\makebox[2cm][c]{99.999} & 
		\makebox[2cm][c]{5} \\
\end{tabular}		  
\end{center}
\captionof{table}{Categorizzazione delle classi di affidabilità dei sistemi informatici. Materiale tratto da: http://bit.ly/2wlGrbn.} 

L'incremento della classe di alta affidabilità è 
un'attività  che richiede costanza e automazione.
I sistemi caratterizzati da alte classi di alta
affidabilità richiedono che essi vengano progettati 
per essere il più possibile autonomi.

\section{Progettazione ed implementazione}
Con la presente sezione descrivo l'architettura di alto
livello del sistema.

\subsection{Vista architetturale}
Per la modellazione di alto livello del sistema ho utilizzato 
lo stile architetturale a Microservizi. Questo stile promuove
la dimensione Y del modello a cubo della scalabilità. Infatti, 
lo stile architetturale a Microservizi porta all'estremo 
il concetto del partizionamento delle funzionalità.
In questo contesto, una funzionalità individua un processo.
Con l'ausilio dei container i microservizi vengono containerizzati.
Per ottenere la scalabilità orizzontalmente è sufficiente aggiungere un nuovo 
processo, copia del processo containerizzato in esecuzione e 
bilanciare il traffico verso l'insieme di processi cosi ottenuto.

In figura che segue, presento i microservizi costituenti il sistema 
da me realizzato.
\newpage
\begin{figure}[htbp]
	\begin{center}
		\includegraphics[height=6cm]{microservizi}
		\caption{Visione di alto livello dell'architettura del sistema.}
	\end{center}
\end{figure}  

Ogni microservizio è responsabile di una sola funzionalità. 
Le figure esagonali sono una convenzione diffusa per 
rappresentare i singoli microservizi costituenti 
un'architettura a Microservizi. Questa notazione ha origine 
nel pattern architetturale "Hexagon Design Pattern", 
promosso da Alistair Cockburn.

%\newpage 

In tabella, per ogni microservizio riporto
una breve descrizione della funzionalità e la tecnologia 
utilizzata per la sua implementazione.  

\begin{center}
	\begin{tabular}
		{l||p{5cm}||p{4cm}}	
		\arrayrulecolor{white}
		\rowcolor{glaucous}	
		Microservizio &  
		\makebox[5cm][c]{Funzionalità) } & 	
		\makebox[4cm][c]{Tecnologia} \\
		\rowcolor{lightcornflowerblue}
		Autorizzazione/Autenticazione & 
		\makebox[5cm][c]{Gestione degli accessi} & 
		\makebox[4cm][c]{Nginx}  \\
		\rowcolor{moonstoneblue}
		Visualizzazione & 
		\makebox[5cm][c]{Gestione delle dashboard} & 
		\makebox[4cm][c]{Kibana} \\
		\rowcolor{lightcornflowerblue}
		Analisi dati & 
		\makebox[5cm][c]{Gestione dei dati in analisi} & 
		\makebox[4cm][c]{Elasticsearch} \\
		\rowcolor{moonstoneblue}
		\textit{Ingesting} & 
		\makebox[5cm][c]{Raccolta ed inserimento dei dati} & 
		\makebox[4cm][c]{Logstash} \\ 
		\rowcolor{lightcornflowerblue}
	\end{tabular}		  
\end{center}
\captionof{table}{Elenco dei microservizi costituenti il sistema.}
%\newpage

\subsection{Flusso operativo}
Il sistema che ho progettato è caratterizzato da un alto tasso di coesione e 
basso accoppiamento. In figura a seguire, con l'ausilio dell'indicazione
numerica, presento i flussi di dati che il sistema deve gestire.

\begin{figure}[htbp]
	\begin{center}
		\includegraphics[height=6.5cm]{flusso}
		\caption{Indicazione dei flussi di dati e delle dipendenze dei microservizi.}
	\end{center}
\end{figure} 

\newpage 

Il sistema è caratterizzato da due flussi distinti.

\begin{enumerate}
	\item \textbf{Flusso 1-2-3}:  \\
	Il seguente flusso rappresenta il flusso inerente
	alla parte esposta all'uso degli utenti.
	\begin{figure}[htbp]
		\begin{center}
			\includegraphics[height=6.5cm]{flusso-principale}
			\caption{Descrizione di alto livello dell'interazione dell'utente con il sistema.}
		\end{center}
	\end{figure} 
    \newpage  
	\item \textbf{Flusso A}: \\
	Un'ulteriore flusso interessa il processo di raccolta ed inserimento dei dati
	nel sistema. Lo strumento ETL (Extract Transform and Load) utilizzato è Logstash.
	Questo flusso non impatta le perfomance complessive del sistema, in quanto 
	la sorgente dati è statica, ha una dimensione fissa. Il flusso è operativo
	solo durante la prima installazione del cluster, successivamente esso 
	è disattivato.
\end{enumerate}

\subsection{Organizzazione del cluster Elasticsearch}

Dal punto di vista applicativo e in contesto WEB, Elasticsearch
può essere interpretato come un backend, ovvero una sorgente di dati. 
Questa componente supporta diverse configurazioni topologiche
di nodi. L'insieme di nodi che non ho trattato sono i nodi per 
la comunicazione inter cluster e distribuiti su zone geografiche diverse.
Un \textit{use case} comune per l'utilizzo di questo tipo di nodi è 
la replica dei dati tra due \textit{Data Center} diversi. 

La progettazione del cluster che ho trattato riguarda una configurazione
di nodi distribuiti nella stessa zona geografica.

In contesto con orchestratore Kubernetes, il cluster minimale è composto 
dal seguente insieme: 1 nodo data, 1 nodo master e 1 nodo client. 
L'ultimo nodo ha il compito di bilanciare il traffico esterno proveniente 
dagli utenti connessi al cluster. Inoltre, i nodi di tipo client mantengono 
una tabella informativa su quale nodo mantiene uno specifico 
frammento di dato. Ho progettato in questo modo l'architettura del sistema 
per agevolare la scalabilità orizzontale delle componenti. Un vincolo forte
per la scalabilità dei nodi master è il seguente: il numero dei nodi 
master deve essere scalato in quantità dispari.

\begin{figure}[htbp]
	\begin{center}
		\includegraphics[height=6cm]{fe-consumers}
		\caption{Architettura del \textit{backend} applicativo.}
	\end{center}
\end{figure}

Il traffico tra i processi Elasticsearch è di due tipi. Il primo traffico 
riguarda quello di servizio ed interno del cluster, è un traffico non visibile 
dall'esterno ed è necessario per coordinare i singoli nodi del cluster, il protocollo 
utilizzato è HTTP e la porta logica è 9300. Sempre attraverso il protocollo HTTP e porta 9200
ho esposto il servizio di Elasticsearch ad ogni altro microservizio consumatore. 
Il traffico entrante è diretto verso il nodo client che in modalità \textit{reverse proxy}
inoltra le richieste ai membri del cluster. Il traffico tra le componenti del 
cluster Elasticsearch è criptato tramite un token. Ogni componente del sistema 
è associata con un utente che ha determinati privilegi. In figura, ho rappresentato 
il token di sicurezza tramite un sole di color giallo. 

L'utente interagisce con il microservizio di gestione delle dashboard.
La tecnologia utilizzata è un applicazione web, che nativamente è 
pensata per comunicare con Elasticsearch. Quest'applicazione, 
essendo priva di stato, è facilmente scalabile in orizzontale.

\begin{figure}[htbp]
	\begin{center}
		\includegraphics[height=6cm]{elk-cluster}
		\caption{Architettura del \textit{frontend} applicativo.}
	\end{center}
\end{figure}

Il sistema espone un unico punto di accesso alle dashboard. 
Ogni utente per accedere al sistema deve essere registrato.
Solo dopo un'autenticazione l'utente può utilizzare il servizio.
Inoltre, per garantire che solo determinati utenti possano accedere al
servizio, il sistema effettua un filtraggio base sugli indirizzi IP.
Infatti, solo gli indirizzi IP nella lista bianca vengono accettati
dal sistema. In caso di altri indirizzi IP, il sistema ignora le richieste.  

Per l'implementazione del microservizio di autenticazione ed autorizzazione 
ho utilizzato Nginx. Nginx è un \textit{load balancer} a livello L7 (Layer 7 OSI), 
bilanciatore del traffico a livello HTTP.

Un'ulteriore questione che ho trattato è il dimensionamento delle componenti 
del sistema. 

\begin{figure}[htbp]
	\begin{center}
		\includegraphics[height=6cm]{dimensionamento}
		\caption{Visione d'insieme per macchina virtuale della richiesta di risorse fisiche.}
	\end{center}
\end{figure}

Ciascuna componente specifica una quantità fissa di risorse. 
Questa quantità non è statica e può variare nel tempo. Durante le 
valutazioni di dimensionamento ho cercato di garantire che il 
totale delle risorse richieste da un insieme di componenti, in una specifica
macchina virtuale, non superi la quantità fisicamente disponibile.
In caso di sovrastima, le macchine virtuali subiranno il OOM (Out of memory).

\newpage   

\subsection{Lo \textit{Storage}}

In contesto containerizzato con orchestratore,
ho utilizzato le risorse di Kubernetes per
isolare le dipendenze delle componenti applicative verso 
il server fisico che gestisce la memorizzazione dei
dati e la loro disponibilità inter macchine virtuali. 

Come \textit{tool} per la condivisone fisica dei dati 
tra le macchine virtuali ho utilizzato il NFS (Networked File System).
Ho valutato anche la tecnologia GlusterFS di RedHat. In seguito a una 
sessione di \textit{brainstorming} con il mio tutor 
abbiamo deciso che la tecnologia per lo \textins{storage distribuito} 
è molto valida e adatta per i container., tuttavia, aggiunge
complessità al progetto. Come conseguenza di questa sessione, 
io e il mio tutor abbiamo confermato l'utilizzo della tecnologia 
NFS.
  
\begin{figure}[htbp]
	\begin{center}
		\includegraphics[height=7cm]{storage}
		\caption{Organizzazione logica dei dati e i livelli di astrazione degli accessi ai dati.}
	\end{center}
\end{figure}

Kubernetes ha il compito di orchestrare i container. Molto spesso, 
Kubernetes schedula un container su una VM e in seguito, in caso 
di non disponibilità di risorse fisiche (CPU, RAM), 
lo scheduler della piattaforma schedula l'esecuzione del 
container su un'altra macchina virtuale. Questo fenomeno 
è comunemente chiamato migrazione.  Il modo con cui Docker e Kubernetes 
esportano i dati dal container sono i volumi. Tuttavia, i volumi 
introducono una dipendenza forte con il host che ospita il container 
in esecuzione. Inoltre, i volumi creati su un host non sono
visibili e raggiungibili da qualunque altro host nella rete.

Kubernetes, tramite gli oggetti per la gestione dei volumi, elimina
l'accoppiamento dei container con l'ambiente virtuale specifico
di esecuzione.

In figura presento i livelli di astrazione usati da Kubernetes 
per garantire la visibilità dei dati tra VM durante la migrazione dei container.
Gli oggetti per la gestione dei dati sono i PersistentVolume e
PersistentVolumeClaims. 

Con i PersistentVolume definisco una capacità di dati che può essere fisicamente 
allocata sul disco fisico di una macchina virtuale. E questa capacità 
è utilizzabile globalmente da tutti i microservizi che hanno accesso. 
Invece, con i PersistentVolumeClaim definisco un sottoinsieme della 
capacità specificata dai PersistentVolume. 
Questo modo di gestire i dati mi permette di rendere indipendente la 
modalità di memorizzazione dei dati su disco dalla modalità di consumo 
dei dati da parte delle componenti containerizzate.
Una simile razionalizzazione dello storage mi offre la possibilità
di portare i dati in qualsiasi ambiente, da un volume NFS a un volume sul
cloud di Amazon oppure Azure ed ecc.

Con l'orchestratore la gestione dei dati è semplice.
In assenza dell'orchestratore la complessità  
è proporzionale al numero dei container da gestire.
Per l'ambiente senza orchestratore ho utilizzato ampiamente
il pattern sidecar applicato allo storage.
La  rappresentazione del pattern segue in figura.   

\begin{figure}[htbp]
	\begin{center}
		\includegraphics[height=6cm]{data-container}
		\caption{Riduzione della dipendenza dei dati dall'infrastruttura fisica in contesto senza orchestratore.}
	\end{center}
\end{figure}

Un alias del sidecar pattern è il \textit{data container}. 
Ho utilizzato questo pattern per progettare la 
l'accesso di scrittura e lettura dei dati a livello
globale senza la necessità di specifica di indirizzi IP. 
Infatti, i container applicativi del sistema 
per accedere ai dati semplicemente 
utilizzano il nome del data container. Dal punto 
di vista dei pattern di progettazione di dettaglio,
un data container è un pattern Singleton. 

%\newpage

\subsection{Il \textit{Networking}}

La rete è un altro aspetto importante nel 
contesto della \textit{light virtualization}.
Kubernetes richiede come vincolo che ad ogni 
Pod venga assegnato un indirizzo IP univoco. 
La necessità di utilizzare un indirizzo IP univoco per 
container è necessaria per evitare la complessità di gestione 
delle porte logiche su cui esporre i servizi.
Inoltre, Kubernetes gestisce il traffico 
tra container tramite regole di routing di basso livello
e gestito in modo automatico. Il traffico del cluster Kubernetes 
è gestito dalla componente Kube-Proxy.
Per assegnare indirizzi univoci ai Pod, ho utilizzato
il plugin WeaveNet. 
Questo strumento mi ha permesso di creare una 
rete \textit{full mesh} tra i container.

\begin{figure}[htbp]
	\begin{center}
		\includegraphics[height=6cm]{networking}
		\caption{Visione di alto livello della topologia di rete.}
	\end{center}
\end{figure}

WeaveNet crea una rete virtuale distribuita su tutte le macchine 
virtuali. L'obbiettivo principale dello strumento è l'assegnazione 
degli indirizzi IP alle risorse Kubernetes \textit{on-demand}. 
La classe degli indirizzi allocabili dalla componente WeaveNet è di tipo A 
(10.1.0.0/16).  
Il prodotto WeaveNet è caratterizzato da un insieme di agenti che cooperano
per gestire il routing del traffico di rete nel modo più ottimale possibile.
Ho gestito il deployment degli agenti WeaveNet tramite l'utilizzo 
della risorsa DeamonSet. Questa risorsa è una estensione della risorsa
Deployment. Tramite la risorsa del DaemonSet, in caso di aggiunta di nodi al 
cluster Kubernetes, il controller del cluster schedulerà dinamicamente 
una istanziazione dell'agente sul nuovo nodo
senza alcun intervento manuale.

Ho utilizzato la ridondanza a livello della rete virtuale per incrementare 
l'alta affidabilità della rete stessa. Infatti, in caso di un problema con 
qualche agente WeaveNet, i pod possono continuare a comunicare con le 
altre componenti in modo trasparente. 

%\newpage

\section{Test}
I test che ho effettuato durante lo stage, per la 
verifica del sistema, sono i test di carico e i test di durata. 
Ho valutato diversi strumenti open source idonei per questa 
tipologia di test. Lo strumento che ho scelto per effettuare i test è JMeter.
La peculiarità di JMeter, che mi ha impressionato, è il variegato 
insieme di plugin e l'insieme di configurazioni che esso supporta.

\begin{center}
	\begin{tabular}
		{l||p{5cm}}	
		\arrayrulecolor{white}
		\rowcolor{glaucous}	
		Configurazione VM & 
		\makebox[5cm][c]{Valore}\\
		\rowcolor{lightcornflowerblue}
		Sistema Operativo &
		\makebox[5cm][c]{CentOS7}\\
		\rowcolor{moonstoneblue}
		Kernel &
		\makebox[5cm][c]{3.10.x86-64}\\
		\rowcolor{lightcornflowerblue}
		CPU(s) &
		\makebox[5cm][c]{4} \\
		\rowcolor{moonstoneblue}
		RAM &
		\makebox[5cm][c]{4GB}\\ 
		\rowcolor{lightcornflowerblue}
	\end{tabular}		  
\end{center}
\captionof{table}{Parametri di configurazione comuni delle macchiane vituali utilizzate durante i test.}

Il taglio di macchine virtuali che ho utilizzato è medio-piccola.
Ho scelto una simile configurazione perché è una categoria molto frequente in cloud.
Questo taglio di macchine sono caratterizzate da un prezzo conveniente. Effettuare 
i test su questo taglio di macchine permette di paragonare i test fatti da altri
con i risultati da me ottenuti.

\subsection{Obiettivi dei test}
Analizzare, progettare ed implementare i test è un forma d'arte.
L'obiettivo principale dei test è lo studio del comportamento
del sistema nel complesso e nell'ambiente containerizzato 
con orchestratore Kubernetes. Per effettuare i test ho fissato 
i limiti di memoria e CPU per ciascuna componente containerizzata. 

Gli obiettivi che ho individuato per i test sono i seguenti:

\begin{itemize}
	\item Studiare il comportamento di Elasticsearch a regime e 
	      sotto carico di lavoro con le configurazioni da me individuate;      
	\item Individuare il punto di rottura del sistema, oltre al quale  
	      il livello di servizio degrada;
	\item Confrontare, a pari di configurazione, il sistema containerizzato 
		  con un sistema containerizzato.
\end{itemize}

\subsection{Metodologia}

In seguito alla preparazione dell'ambiente di test, 
ho individuato degli scenari possibili di test. 
Tramite uno scenario cerco di simulare l'attività 
dell'utente con il sistema. L'utente virtuale deve 
effettuare una serie di richieste di complessità sempre
crescente. Le richieste riguardano la visualizzazione 
delle dashboard. Durante le prime visualizzazioni
l'utente virtuale visualizza i singoli elementi 
alla base di una dashboard complessa. E successivamente 
tramite aggregazione l'utente effettua visualizzazioni
sempre più complesse e numerose.
Con JMeter ho codificato degli script per i scenari di test. 
Un esempio di script, che allego in figura per il test di durata, 
ha lo scopo di simulare 100 utenti che interagiscono con il sistema 
in un intervallo temporale di un'ora. 

\begin{figure}[htbp]
	\begin{center}
		\includegraphics[height=6cm]{jmeter}
		\caption{Script JMeter per i test di durata con 100 utenti, 1h di durata.}
	\end{center}
\end{figure} 

La configurazione utilizzata per lo scenario del test 
è come nella tabella seguente.

\begin{center}
	\begin{tabular}
		{l||p{5cm}}	
		\arrayrulecolor{white}
		\rowcolor{glaucous}	
		Parametro di configurazione &  
		\makebox[5cm][c]{Valore) } \\
		\rowcolor{lightcornflowerblue}
		Utenti & 
		\makebox[5cm][c]{100} \\
		\rowcolor{moonstoneblue}
		Tempo d'inizializzazione & 
		\makebox[5cm][c]{20 minuti} \\
		\rowcolor{lightcornflowerblue}
		Fasi di inizializzazione & 
		\makebox[5cm][c]{10}  \\
		\rowcolor{moonstoneblue}
		Durata test & 
		\makebox[5cm][c]{60 minuti}\\ 
		\rowcolor{lightcornflowerblue}
	\end{tabular}		  
\end{center}
\captionof{table}{Paramentri di configurazione di uno scenario di esempio del test di durata.}

Come output, per ciascun test in parte ho generato un resoconto finale.
Per la creazione dei report, ho progettato ed implementato un processo automatico.
Ho scelto le ore notturne per effettuare i test e generare i report per i seguenti
motivi: 
\begin{itemize}
	\item Utilizzare le ore di lavoro regolari per fare analisi e studio dei dati;
	\item Impiegare i server fisici di IKS al massimo durante le ore notturne;
	\item Ridurre al minimo gli impatti delle eventuali interferenze sulle macchine 
	      virtuali da me utilizzate.  
\end{itemize}


\subsection{Risultati}
Durante i test, ho appreso che con specifiche configurazioni di memoria, CPU e 
topologia del cluster Elasticsearch il sistema è più responsivo. 
Ottenere una configurazione del sistema con specifici limiti sulle
non è stato facile. Questa ricerca ha rafforzato la mia confidenza 
con l'amministrazione del cluster Kubernetes.
 
In figura a seguire, presento il comportamento dei tempi di risposta in 
relazione con il numero di utenti attivi nel sistema. 

\begin{figure}[htbp]
	\begin{center}
		\includegraphics[height=5cm]{flotTimesVsThreads}
		\caption{Andamento dei tempi di risposta all'aumentare del numero degli utenti.}
	\end{center}
\end{figure}

Qui, il numero dei thread rappresentano il numero di utenti attivi nel sistema. 
I dati mostrano che all'aumentare del numero degli utenti, per le tre 
attività di visualizzazione di dashboard, i tempi di risposta sono 
sempre al di sotto dei 9 secondi. Questo è un buon risultato, assumendo 
come limiti di attesa i 10 secondi per ottenere la visualizzazione di una 
pagina, durante l'attività di carico del sistema. Ho recuperato
questa immagine da uno dei report che il processo di test ha generato.

\newpage
\section{Monitoraggio}

Il monitoraggio delle applicazioni e dell'infrastruttura 
è un aspetto molto importante. Sia Docker sia Kubernetes 
non offrono soluzioni native di monitoraggio. I comuni 
strumenti di monitoraggio, orientati a macchine virtuali, 
non sono più idonei. Le macchine virtuali hanno un tempo di
accensione di un ordine di grandezza superiore rispetto 
ai container. Se le macchine virtuali operano nell'ordine 
dei minuti, allora i container operano nell'arco dei secondi. 

Visto che non esiste una soluzione nativa per l'esportazione delle
metriche in Docker, ho utilizzato la componente cAdvisor che
esporta i dettagli da monitorare nel comodissimo formatto JSON. 
Definito il modo per l'estrapolazione delle informazioni 
sui container, ho integrato cAdvisor con InfluxDB, un \textit{timeseries database}, 
e Graphana, una componente di visualizzazione. Quest'ultima offre 
un familiare DSL (\textit{Domain Specific Language}) 
in stile SQL (\textit{Search Query Language}).  

Un'ulteriore componente che ho utilizzato per il monitoraggio 
dell'infrastruttura è la \textit{dashboard} di Kubernetes.
Oltre al monitoraggio, la dashboard nativa di Kubernetes
offre la possibilità di scrivere nuovi manifest, editare 
e cancellare quelli esistenti. Oltre alla creazione di risorse
è possibile visionare i log per ciascun container da un unico 
punto di accesso. La dashboard centralizza la gestione del cluster. 

In conclusione di stage, mi sono reso conto che la visione 
d'insieme del cluster e del sistema realizzato è poco chiara. 
Uno svantaggio di un sistema containerizzato è la mancanza 
della rappresentazione grafica delle relazioni tra le componenti.
Questo è dovuto alla natura distribuita dei microservizi. 
Con questo fatto in mente, ho ridotto la difficoltà di gestione
tramite uno strumento open source. Il nome di questo strumento 
è Weave Scope. 

\begin{figure}[htbp]
	\begin{center}
		\includegraphics[height=7.5cm]{pods}
		\caption{Grafo dei Pod a supporto del cluster Kubernetes.}
	\end{center}
\end{figure} 
\newpage
Per lo studio del consumo delle risorse, il prodotto di Weaveworks
offre una vista dello stato delle risorse sul server. In figura 
illustro qual è lo stato delle risorse del nodo master del cluster K8s.
\begin{figure}[htbp]
	\begin{center}
		\includegraphics[height=5cm]{resource-consumption}
		\caption{Vista del consumo di risorse su un nodo del cluster Kubernetes.}
	\end{center}
\end{figure}

In conclusione, illustro la visione completa del sistema 
dal punto di vista delle componenti. Ogni componente 
è presentata in versione ridondata.
\newpage
\begin{figure}[htbp]
	\begin{center}
		\includegraphics[height=7cm]{panoramica-sistema}
		\caption{Visione completa delle componenti del sistema e delle loro relazioni.}
	\end{center}
\end{figure}
\newpage
            
% !TEX encoding = UTF-8
% !TEX TS-program = pdflatex
% !TEX root = ../tesi.tex
% !TEX spellcheck = it-IT

\chapter{Valutazioni retrospettive}
\label{cap:valutazioni-retrospettive}
\section{Obiettivi raggiunti}
\section{Problematiche riscontrate}

%Svolgendo quest'attività di formazione ho riscontrato 
%alcuni comportamenti anomali del cluster Elasticsearch. 
%Spesso, in seguito ad aggiunte oppure rimozioni di nodi a 
%caldo sul cluster Elasticsearch, 
%ottenevo degli insiemi di cluster disgiunti e i nodi 
%membri di un insieme erano incapaci di comunicare con i nodi 
%dell'altro insieme. In seguito a qualche 
%approfondimento, ho scoperto che il problema era dovuto
%a una mia mal organizzazione della topologia del cluster 
%e suddivisione dei ruoli tra i nodi Elasticsearch. 
%Infatti, spesso i nodi non riuscivano ad ottenere 
%un consenso di maggioranza a favore del nuovo leader del gruppo, 
%in seguito le operazioni di modifica della topologia del cluster.

%La causa del problema è la seguente: aggiungendo oppure rimuovendo un 
%nodo al cluster, il quorum (insieme di membri costituenti un gruppo) 
%risultante incomincia un'attività di elezione del nuovo leader del gruppo. 
%Gestendo erroneamente i ruoli interni del cluster, la votazione per 
%eleggere un nuovo leader non porta un risultato univoco e i nodi formano 
%sottoinsiemi disgiunti. Così, ciascun insieme elegge un proprio leader
%tramite il voto della maggioranza.  
%Elasticsearch utilizza il paradigma Master-Slave di calcolo e 
%offre diverse tipologie di nodi. 


\section{Bilancio formativo}
\section{Valutazione critica del Corso di Laurea}
\newpage
             
%\appendix                               
%% !TEX encoding = UTF-8
% !TEX TS-program = pdflatex
% !TEX root = ../tesi.tex
% !TEX spellcheck = it-IT

%**************************************************************
\chapter{Appendice A}
%**************************************************************
% \epigraph{Citazione}{Autore della citazione}

\section{Machine Learning}

\section{Stile architetturale a microservizi}

\section{Containerizzazione}

\section{Cloud}

             

%**************************************************************
% Materiale finale
%**************************************************************
\backmatter
\printglossaries

\input{inizio-fine/bibliografia}
\end{document}