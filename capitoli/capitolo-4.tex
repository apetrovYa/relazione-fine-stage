% !TEX encoding = UTF-8
% !TEX TS-program = pdflatex
% !TEX root = ../tesi.tex
% !TEX spellcheck = it-IT

\chapter{Valutazioni retrospettive}
\label{cap:valutazioni-retrospettive}
\section{Obiettivi raggiunti}
\section{Problematiche riscontrate}

%Svolgendo quest'attività di formazione ho riscontrato 
%alcuni comportamenti anomali del cluster Elasticsearch. 
%Spesso, in seguito ad aggiunte oppure rimozioni di nodi a 
%caldo sul cluster Elasticsearch, 
%ottenevo degli insiemi di cluster disgiunti e i nodi 
%membri di un insieme erano incapaci di comunicare con i nodi 
%dell'altro insieme. In seguito a qualche 
%approfondimento, ho scoperto che il problema era dovuto
%a una mia mal organizzazione della topologia del cluster 
%e suddivisione dei ruoli tra i nodi Elasticsearch. 
%Infatti, spesso i nodi non riuscivano ad ottenere 
%un consenso di maggioranza a favore del nuovo leader del gruppo, 
%in seguito le operazioni di modifica della topologia del cluster.

%La causa del problema è la seguente: aggiungendo oppure rimuovendo un 
%nodo al cluster, il quorum (insieme di membri costituenti un gruppo) 
%risultante incomincia un'attività di elezione del nuovo leader del gruppo. 
%Gestendo erroneamente i ruoli interni del cluster, la votazione per 
%eleggere un nuovo leader non porta un risultato univoco e i nodi formano 
%sottoinsiemi disgiunti. Così, ciascun insieme elegge un proprio leader
%tramite il voto della maggioranza.  
%Elasticsearch utilizza il paradigma Master-Slave di calcolo e 
%offre diverse tipologie di nodi. 


\section{Bilancio formativo}
\section{Valutazione critica del Corso di Laurea}
\newpage
