% !TEX encoding = UTF-8
% !TEX TS-program = pdflatex
% !TEX root = ../tesi.tex
% !TEX spellcheck = it-IT

%**************************************************************
\chapter{Lo svolgimento dello stage}
\label{cap:svolgimento-stage}
%**************************************************************
%\intro{Aggiungere abstract}\\
%\textbf{TODO: Aggiungere sintesi al capitolo}\\


%**************************************************************
\section{Metodo di lavoro}




\section{Attività di formazione}

\section{Analisi dei requisiti}
\subsection{Classificazione dei requisiti}
\subsection{Requisiti}
\subsubsection{Funzionali}
\subsubsection{Non funzionali}


%**************************************************************
\section{Progettazione e realizzazione}

\subsection{Scelte progettuali}

\subsection{Visione architetturale a microservizi}

\subsection{Codifica}

\subsection{Test}

\subsubsection{Test di carico}
\subsubsection{Test di durata}








%\begin{risk}{Performance del simulatore hardware}
%    \riskdescription{le performance del simulatore hardware e la comunicazione con questo potrebbero risultare lenti o non abbastanza buoni da causare il fallimento dei test}
%    \risksolution{coinvolgimento del responsabile a capo del progetto relativo il simulatore hardware}
%    \label{risk:hardware-simulator} 
%\end{risk}

