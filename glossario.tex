
%**************************************************************
% Acronimi
%**************************************************************
\renewcommand{\acronymname}{Acronimi e abbreviazioni}

\newacronym[description={\glslink{apig}{Application Program Interface}}]
    {api}{API}{Application Program Interface}


\newacronym[description={\glslink{iksg}{Information and Knowledge Supply}}]
	{iks}{IKS}{Information and Knowledge Supply}

\newacronym[description={\glslink{ictg}{Information and Communication Technology}}]
	{ict}{ICT}{Information and Communication Technology}

\newacronym[description={\glslink{itg}{Information Technology}}]
	{it}{IT}{Information Technology}


%**************************************************************
% Glossario
%**************************************************************
\renewcommand{\glossaryname}{Glossario}

\newglossaryentry{framework}{
	name=\glslink{framework}{Framework},
	text=Framework,
	sort=framework,
	description={Architettura logica di supporto su cui un software
		può essere progettato e realizzato, spesso facilitandone lo sviluppo da parte del
		programmatore. La sua funzione è quella di creare una infrastruttura generale,
		lasciando al programmatore il contenuto vero e proprio dell’applicazione.
		}
}


\newglossaryentry{patching}{
	name=\glslink{patching}{Patching},
	text=Patching,
	sort=patching,
	description={
	Applicare una patch, porzione di software progettata per aggiornare o migliorare un programma. Una patch permette di risolvere vulnerabilità di sicurezza e altri BugFix di un applicativo sviluppato.
	}
}
\newglossaryentry{agile}{
	name=\glslink{agile}{Agile},
	text=Agile (metodologia),
	sort=agile,
	description={
	Metodo per lo sviluppo del software che coinvolge quanto pi`u possibile il committente, ottenendo in tal modo una elevata reattivit`a alle sue richieste.
	}
}
\newglossaryentry{repository}{
	name=\glslink{repository}{Repository},
	text=Repository,
	sort=repository,
	description={
	Ambiente di un sistema informativo, in cui vengono gestiti
	i metadati, attraverso tabelle relazionali; l’insieme di tabelle, regole e motori
	di calcolo tramite cui si gestiscono i metadati prende il nome di metabase.
	}
}
\newglossaryentry{deployment}{
	name=\glslink{deployment}{Deployment},
	text=Deployment,
	sort=deployment,
	description={
	Consegna o rilascio al cliente, con relativa installazione e messa
	in funzione o esercizio, di una applicazione o di un sistema software tipicamente
	all’interno di un sistema informatico aziendale.
	}
}

\newglossaryentry{cloud}{
	name=\glslink{cloud}{Cloud},
	text=Cloud,
	sort=cloud,
	description={
	Paradigma di erogazione di risorse informatiche, come
	l’archiviazione, l’elaborazione o la trasmissione di dati, caratterizzato dalla
	disponibilità on demand attraverso Internet a partire da un insieme di risorse
	preesistenti e configurabili.
	}
}
%\newglossaryentry{}
%\newglossaryentry{}
%\newglossaryentry{}
%\newglossaryentry{}
%\newglossaryentry{}
%\newglossaryentry{}
%\newglossaryentry{}

\newglossaryentry{apig}
{
    name=\glslink{api}{API},
    text=Application Program Interface,
    sort=api,
    description={Il termine \emph{Application Programming Interface API} (ing. interfaccia di programmazione di un'applicazione) si indica ogni insieme di procedure disponibili al programmatore, di solito raggruppate a formare un set di strumenti specifici per l'espletamento di un determinato compito all'interno di un certo programma. La finalità è ottenere un'astrazione, di solito tra l'hardware e il programmatore o tra software a basso e quello ad alto livello semplificando così il lavoro di programmazione}
}

\newglossaryentry{ictg}
{
	name=\glslink{ict}{ICT},
	text=Information and Communication Technology,
	sort=ict,
	description={insieme di metodi e tecnologie che implementano i sistemi di trasmissione, ricezione e elaborazione di informazioni.}
}

\newglossaryentry{itg}
{
	name=\glslink{it}{IT},
	text=Information Technology,
	sort=it,
	description={utilizzo di qualsiasi tecnologia di calcolo per offrire servizio di memorizzazione, reti per creare, processare, memorizzare e mettere in sicurezza ogni forma immaginabile di dato elettronico.
    }
}



\newglossaryentry{private equity}{
	name=\glslink{private equity}{Private equity},
	text=Private equity,
	sort=private,
	description={
	Da definire
	}
}

\newglossaryentry{IKS}{
	name=\glslink{iks}{IKS},
	text=IKS,
	sort=iks,
	description={
		IL nome dell'azienda è un acronimo inglese che ne determina lo spirito e la filosofia. 
	}
}

