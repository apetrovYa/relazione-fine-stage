%**************************************************************
% file contenente le impostazioni della tesi
%**************************************************************

%**************************************************************
% Frontespizio
%**************************************************************
\newcommand{\myName}{Andrei Petrov}                                    % autore
\newcommand{\myTitle}{Containerizzazione di Malware Dashboard}                    
\newcommand{\myDegree}{Relazione finale di stage}                % tipo di tesi
\newcommand{\myUni}{Università degli Studi di Padova}           % università
\newcommand{\myFaculty}{Corso di Laurea in Informatica}         % facoltà
\newcommand{\myDepartment}{Dipartimento di Matematica}          % dipartimento
\newcommand{\myProf}{Tullio Vardanega}                                % relatore
\newcommand{\myLocation}{Padova}                                % dove
\newcommand{\myAA}{2016-2017}                                   % anno accademico
\newcommand{\myTime}{ Luglio - 2017}                                  % quando


%**************************************************************
% Impostazioni di impaginazione
% see: http://wwwcdf.pd.infn.it/AppuntiLinux/a2547.htm
%**************************************************************

\setlength{\parindent}{14pt}   % larghezza rientro della prima riga
\setlength{\parskip}{0pt}   % distanza tra i paragrafi


%**************************************************************
% Impostazioni di biblatex
%**************************************************************
\bibliography{bibliografia} % database di biblatex 

\defbibheading{bibliography}
{
    \cleardoublepage
    \phantomsection 
    \addcontentsline{toc}{chapter}{\bibname}
    \chapter*{\bibname\markboth{\bibname}{\bibname}}
}

\setlength\bibitemsep{1.5\itemsep} % spazio tra entry

\DeclareBibliographyCategory{opere}
\DeclareBibliographyCategory{web}

\addtocategory{opere}{womak:lean-thinking}
\addtocategory{opere}{baier:getting-started-k8s}
\addtocategory{opere}{vohra:k8s-micro-docker}
\addtocategory{opere}{turnbull:docker-book}
\addtocategory{opere}{gormlery:es}
\addtocategory{opere}{kuc:es}

\addtocategory{web}{site:k8s}
\addtocategory{web}{site:docker}
\addtocategory{web}{site:es}
\addtocategory{web}{site:logstash}
\addtocategory{web}{site:kibana}
\addtocategory{web}{site:nginx}
\addtocategory{web}{site:linux}
\addtocategory{web}{site:micro}
\addtocategory{web}{site:micro-fawler}
\addtocategory{web}{site:micro-wiki}
\addtocategory{web}{site:cloud-wiki}

\defbibheading{opere}{\section*{Bibliografia}}
\defbibheading{web}{\section*{Sitografia}}


%**************************************************************
% Impostazioni di caption
%**************************************************************
\captionsetup{
    tableposition=top,
    figureposition=bottom,
    font=small,
    format=hang,
    labelfont=bf
}

%**************************************************************
% Impostazioni di glossaries
%**************************************************************

%**************************************************************
% Acronimi
%**************************************************************
\renewcommand{\acronymname}{Acronimi e abbreviazioni}

\newacronym[description={\glslink{apig}{Application Program Interface}}]
    {api}{API}{Application Program Interface}


\newacronym[description={\glslink{iksg}{Information and Knowledge Supply}}]
	{iks}{IKS}{Information and Knowledge Supply}

\newacronym[description={\glslink{ictg}{Information and Communication Technology}}]
	{ict}{ICT}{Information and Communication Technology}

\newacronym[description={\glslink{itg}{Information Technology}}]
	{it}{IT}{Information Technology}


%**************************************************************
% Glossario
%**************************************************************
\renewcommand{\glossaryname}{Glossario}

\newglossaryentry{framework}{
	name=\glslink{framework}{Framework},
	text=Framework,
	sort=framework,
	description={Architettura logica di supporto su cui un software
		può essere progettato e realizzato, spesso facilitandone lo sviluppo da parte del
		programmatore. La sua funzione è quella di creare una infrastruttura generale,
		lasciando al programmatore il contenuto vero e proprio dell’applicazione.
		}
}


\newglossaryentry{patching}{
	name=\glslink{patching}{Patching},
	text=Patching,
	sort=patching,
	description={
	Applicare una patch, porzione di software progettata per aggiornare o migliorare un programma. Una patch permette di risolvere vulnerabilità di sicurezza e altri BugFix di un applicativo sviluppato.
	}
}
\newglossaryentry{agile}{
	name=\glslink{agile}{Agile},
	text=Agile (metodologia),
	sort=agile,
	description={
	Metodo per lo sviluppo del software che coinvolge quanto pi`u possibile il committente, ottenendo in tal modo una elevata reattivit`a alle sue richieste.
	}
}
\newglossaryentry{repository}{
	name=\glslink{repository}{Repository},
	text=Repository,
	sort=repository,
	description={
	Ambiente di un sistema informativo, in cui vengono gestiti
	i metadati, attraverso tabelle relazionali; l’insieme di tabelle, regole e motori
	di calcolo tramite cui si gestiscono i metadati prende il nome di metabase.
	}
}
\newglossaryentry{deployment}{
	name=\glslink{deployment}{Deployment},
	text=Deployment,
	sort=deployment,
	description={
	Consegna o rilascio al cliente, con relativa installazione e messa
	in funzione o esercizio, di una applicazione o di un sistema software tipicamente
	all’interno di un sistema informatico aziendale.
	}
}

\newglossaryentry{cloud}{
	name=\glslink{cloud}{Cloud},
	text=Cloud,
	sort=cloud,
	description={
	Paradigma di erogazione di risorse informatiche, come
	l’archiviazione, l’elaborazione o la trasmissione di dati, caratterizzato dalla
	disponibilità on demand attraverso Internet a partire da un insieme di risorse
	preesistenti e configurabili.
	}
}
%\newglossaryentry{}
%\newglossaryentry{}
%\newglossaryentry{}
%\newglossaryentry{}
%\newglossaryentry{}
%\newglossaryentry{}
%\newglossaryentry{}

\newglossaryentry{apig}
{
    name=\glslink{api}{API},
    text=Application Program Interface,
    sort=api,
    description={Il termine \emph{Application Programming Interface API} (ing. interfaccia di programmazione di un'applicazione) si indica ogni insieme di procedure disponibili al programmatore, di solito raggruppate a formare un set di strumenti specifici per l'espletamento di un determinato compito all'interno di un certo programma. La finalità è ottenere un'astrazione, di solito tra l'hardware e il programmatore o tra software a basso e quello ad alto livello semplificando così il lavoro di programmazione}
}

\newglossaryentry{ictg}
{
	name=\glslink{ict}{ICT},
	text=Information and Communication Technology,
	sort=ict,
	description={insieme di metodi e tecnologie che implementano i sistemi di trasmissione, ricezione e elaborazione di informazioni.}
}

\newglossaryentry{itg}
{
	name=\glslink{it}{IT},
	text=Information Technology,
	sort=it,
	description={utilizzo di qualsiasi tecnologia di calcolo per offrire servizio di memorizzazione, reti per creare, processare, memorizzare e mettere in sicurezza ogni forma immaginabile di dato elettronico.
    }
}



\newglossaryentry{private equity}{
	name=\glslink{private equity}{Private equity},
	text=Private equity,
	sort=private,
	description={
	Da definire
	}
}

\newglossaryentry{IKS}{
	name=\glslink{iks}{IKS},
	text=IKS,
	sort=iks,
	description={
		IL nome dell'azienda è un acronimo inglese che ne determina lo spirito e la filosofia. 
	}
}

 % database di termini
\makeglossaries


%**************************************************************
% Impostazioni di graphicx
%**************************************************************
\graphicspath{{immagini/}} % cartella dove sono riposte le immagini


%**************************************************************
% Impostazioni di hyperref
%**************************************************************
\hypersetup{
    %hyperfootnotes=false,
    %pdfpagelabels,
    %draft,	% = elimina tutti i link (utile per stampe in bianco e nero)
    colorlinks=true,
    linktocpage=true,
    pdfstartpage=1,
    pdfstartview=FitV,
    % decommenta la riga seguente per avere link in nero (per esempio per la stampa in bianco e nero)
    %colorlinks=false, linktocpage=false, pdfborder={0 0 0}, pdfstartpage=1, pdfstartview=FitV,
    breaklinks=true,
    pdfpagemode=UseNone,
    pageanchor=true,
    pdfpagemode=UseOutlines,
    plainpages=false,
    bookmarksnumbered,
    bookmarksopen=true,
    bookmarksopenlevel=1,
    hypertexnames=true,
    pdfhighlight=/O,
    %nesting=true,
    %frenchlinks,
    urlcolor=webbrown,
    linkcolor=RoyalBlue,
    citecolor=webgreen,
    %pagecolor=RoyalBlue,
    %urlcolor=Black, linkcolor=Black, citecolor=Black, %pagecolor=Black,
    pdftitle={\myTitle},
    pdfauthor={\textcopyright\ \myName, \myUni, \myFaculty},
    pdfsubject={},
    pdfkeywords={},
    pdfcreator={pdfLaTeX},
    pdfproducer={LaTeX}
}

%**************************************************************
% Impostazioni di itemize
%**************************************************************
\renewcommand{\labelitemi}{$\ast$}

%\renewcommand{\labelitemi}{$\bullet$}
%\renewcommand{\labelitemii}{$\cdot$}
%\renewcommand{\labelitemiii}{$\diamond$}
%\renewcommand{\labelitemiv}{$\ast$}


%**************************************************************
% Impostazioni di listings
%**************************************************************
\lstset{
    language=[LaTeX]Tex,%C++,
    keywordstyle=\color{RoyalBlue}, %\bfseries,
    basicstyle=\small\ttfamily,
    %identifierstyle=\color{NavyBlue},
    commentstyle=\color{Green}\ttfamily,
    stringstyle=\rmfamily,
    numbers=none, %left,%
    numberstyle=\scriptsize, %\tiny
    stepnumber=5,
    numbersep=8pt,
    showstringspaces=false,
    breaklines=true,
    frameround=ftff,
    frame=single
} 


%**************************************************************
% Impostazioni di xcolor
%**************************************************************
\definecolor{webgreen}{rgb}{0,.5,0}
\definecolor{webbrown}{rgb}{.6,0,0}


%**************************************************************
% Altro
%**************************************************************

\newcommand{\omissis}{[\dots\negthinspace]} % produce [...]

% eccezioni all'algoritmo di sillabazione
\hyphenation
{
    ma-cro-istru-zio-ne
    gi-ral-din
}

\newcommand{\sectionname}{sezione}
\addto\captionsitalian{\renewcommand{\figurename}{Figura}
                       \renewcommand{\tablename}{Tabella}}

\newcommand{\glsfirstoccur}{\ap{{[g]}}}

\newcommand{\intro}[1]{\emph{\textsf{#1}}}

%**************************************************************
% Environment per ``rischi''
%**************************************************************
\newcounter{riskcounter}                % define a counter
\setcounter{riskcounter}{0}             % set the counter to some initial value

%%%% Parameters
% #1: Title
\newenvironment{risk}[1]{
    \refstepcounter{riskcounter}        % increment counter
    \par \noindent                      % start new paragraph
    \textbf{\arabic{riskcounter}. #1}   % display the title before the 
                                        % content of the environment is displayed 
}{
    \par\medskip
}

\newcommand{\riskname}{Rischio}

\newcommand{\riskdescription}[1]{\textbf{\\Descrizione:} #1.}

\newcommand{\risksolution}[1]{\textbf{\\Soluzione:} #1.}

%**************************************************************
% Environment per ``use case''
%**************************************************************
\newcounter{usecasecounter}             % define a counter
\setcounter{usecasecounter}{0}          % set the counter to some initial value

%%%% Parameters
% #1: ID
% #2: Nome
\newenvironment{usecase}[2]{
    \renewcommand{\theusecasecounter}{\usecasename #1}  % this is where the display of 
                                                        % the counter is overwritten/modified
    \refstepcounter{usecasecounter}             % increment counter
    \vspace{10pt}
    \par \noindent                              % start new paragraph
    {\large \textbf{\usecasename #1: #2}}       % display the title before the 
                                                % content of the environment is displayed 
    \medskip
}{
    \medskip
}

\newcommand{\usecasename}{UC}

\newcommand{\usecaseactors}[1]{\textbf{\\Attori Principali:} #1. \vspace{4pt}}
\newcommand{\usecasepre}[1]{\textbf{\\Precondizioni:} #1. \vspace{4pt}}
\newcommand{\usecasedesc}[1]{\textbf{\\Descrizione:} #1. \vspace{4pt}}
\newcommand{\usecasepost}[1]{\textbf{\\Postcondizioni:} #1. \vspace{4pt}}
\newcommand{\usecasealt}[1]{\textbf{\\Scenario Alternativo:} #1. \vspace{4pt}}

%**************************************************************
% Environment per ``namespace description''
%**************************************************************

\newenvironment{namespacedesc}{
    \vspace{10pt}
    \par \noindent                              % start new paragraph
    \begin{description} 
}{
    \end{description}
    \medskip
}

\newcommand{\classdesc}[2]{\item[\textbf{#1:}] #2}